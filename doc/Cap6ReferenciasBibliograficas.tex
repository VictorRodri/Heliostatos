\section{Referencias bibliográficas.}

[1] Gstriatum (n. d.). Energías renovables. [Mensaje en un blog]. Recuperado el 10 de febrero de 2.018 de: http://www.gstriatum.com/energiasolar/blog/

[2] EcuRed (n. d.). Biblioteca Pública Municipal Sandino. Recuperado el 10 de febrero de 2.018 de: https://www.ecured.cu/Biblioteca\_P\%C3\%BAblica\_Municipal\_Sandino

[5] Oepm (2.011). Helióstato con sensor de reflexión. Recuperado el 11 de abril de 2.018 de: http://www.economiadelaenergia.com/2011/05/heliostato-con-sensor-de-reflexion/

[8] Lucía Rincón (2.013). Funcionamiento de la central solar. Recuperado el 6 de mayo de 2.018 de: https://sites.google.com/site/energiasolar3b/funcionamiento-de-la-central-solar

[10] igualada.institucio.org (n. d.). N. d. Recuperado el 15 de junio de 2.018 de: https://igualada.institucio.org/ca/.../centraterm.htm

[11] Soliclima (n. d.). Energía termoeléctrica. Recuperado el 15 de junio de 2.018 de: http://www.soliclima.com/termoelectrica.html

[12] Claudio (2.014). Centrales solares: generar energía eléctrica con el Sol. Recuperado el 15 de junio de 2.018 de: https://historiaybiografias.com/central\_solar/

[14] Jonathan Préstamo Rodríguez (2.014). PS10: La megatorre sevillana de la energía solar (y cómo funciona). Recuperado el 16 de junio de 2.018 de: http://www.teknoplof.com/tag/heliostato/

[29] Valenticampderros A. (11 marzo 2.012). PS10 y PS20. [Mensaje en un blog]. Recuperado el 18 de junio de 2.018 de: https://themorningstarg2.wordpress.com/tag/heliostatos/

[30] Pons, C. (26 febrero 2.012). Gemasolar. [Mensaje en un blog]. Recuperado el 18 de junio de 2.018 de: https://themorningstarg2.wordpress.com/tag/heliostatos/

[33] OpenCV (2.018). OpenCV. Recuperado el 20 de junio de 2.018 de: http://opencv.org/

[34] SAL (2.015). Infraestructura. Recuperado el 20 de junio de 2.018 de: http://www.hpca.ual.es/es/infraestructura.

[35] IEEE Xplore (2.018). IEEE Xplore. Recuperado el 20 de junio de 2.018 de: https://ieeexplore.ieee.org/Xplore/home.jsp

[36] SourgeForce.net (2.018). N. d. Recuperado el 17 de julio de 2.018 de: https://sourceforge.net/projects/opencv/

[37] Ubaa.net (n. d.). OpenCV - Processing and Java Library. Recuperado el 17 de julio de 2.018 de: http://ubaa.net/shared/processing/opencv/

[38] Yahoo (n. d.). OpenCV en Español. Recuperado el 17 de julio de 2.018 de: https://groups.yahoo.com/neo/groups/OpenCVenEspanol/info

[39] Bartolomé Sintes Marco (2.018). Historia de Python. Recuperado el 17 de julio de 2.018 de: http://www.mclibre.org/consultar/python/otros/historia.html

[41] Wikipedia (25 de septiembre de 2.018). Python. Recuperado el 27 de septiembre de 2.018 de: https://es.wikipedia.org/wiki/Python

[42] BeJob (18 de septiembre de 2.016). 7 razones para programar en Python. Recuperado el 27 de septiembre de 2.018 de: https://www.bejob.com/7-razones-para-programar-en-python/

[43] Google (n. d.). RE<C: Heliostat Project Overview. Recuperado el 5 de octubre de 2.018 de: https://www.google.org/pdfs/google\_heliostat\_project.pdf

[45] Wikipedia (2.018). Instalación de energía solar Ivanpah. Recuperado el 16 de octubre de 2.018 de: https://en.wikipedia.org/wiki/Ivanpah\_Solar\_Power\_Facility

[46] Wikipedia (2.018). File: PS10 solar power tower.jpg. Recuperado el 27 de octubre de 2.018 de: https://en.wikipedia.org/wiki/File:PS10\_solar\_power\_tower.jpg

[47] PSA (n. d.). Instalaciones de la PSA. Recuperado el 27 de octubre de 2.018 de: http://www.psa.es/es/instalaciones/images/cesa.jpg

[48] PSA (n. d.). Instalaciones de la PSA. Recuperado el 27 de octubre de 2.018 de: http://www.psa.es/es/instalaciones/images/crs.jpg

[49] PSA (n. d.). La instalación CESA-1 de 7MWt. Recuperado el 27 de octubre de 2.018 de: http://www.psa.es/es/instalaciones/receptor/cesa1.php

[50] PSA (n. d.). La instalación SSPS-CRS de 2,5 MWt. Recuperado el 27 de octubre de 2.018 de: http://www.psa.es/es/instalaciones/receptor/crs.php