\documentclass[12pt]{article}

\usepackage[utf8x]{inputenc}
\usepackage[spanish,es-noshorthands]{babel}

\usepackage{amssymb,amsmath,amsthm,amsfonts}
\usepackage{calc}
\usepackage{graphicx}
\usepackage{subfigure}
\usepackage{gensymb}
\usepackage{natbib}
\usepackage{url}
\usepackage[utf8x]{inputenc}
\usepackage{amsmath}
\usepackage{graphicx}
\graphicspath{{images/}}
\usepackage{parskip}
\usepackage{fancyhdr}
\usepackage{vmargin}
\usepackage{verbatim}
\setmarginsrb{3 cm}{2.5 cm}{3 cm}{2.5 cm}{1 cm}{1.5 cm}{1 cm}{1.5 cm}

\title{Tu titulo}					% Titulo
\author{Tu nombre}					% Autor
\date{\today}						% Fecha


\makeatletter
\let\thetitle\@title
\let\theauthor\@author
\let\thedate\@date
\makeatother

\pagestyle{fancy}
\fancyhf{}
\rhead{\theauthor}
\lhead{\thetitle}
\cfoot{\thepage}

\begin{document}

\tableofcontents
\pagebreak

%%%%%%%%%%%%%%%%%%%%%%%%%%%%%%%%%%%%%%%%%%%%%%%%%%%%%%%%%%%%%%%%%%%%%%%%%%%%%%%%%%%%%%%%%

\section{Descripción de los requerimientos/requisitos.}

En la elaboración de este TFG, se ha creado además un programa de ordenador en lenguaje de programación Python. Para ello, se requerían de los siguientes elementos:

Ordenador personal.
Tener instalados los siguientes programas en el sistema:
‘IDLE (Python 3.6 64-bit)’, concretamente la versión ‘3.6.4’.
‘Git Bash’, versión 2.7.7 para sistemas de 64 bits.
Agregar unas variables de entorno en el sistema, explicado previamente en este informe.

\end{document}