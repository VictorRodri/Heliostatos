\documentclass[12pt]{article}

\usepackage[utf8x]{inputenc}
\usepackage[spanish,es-noshorthands]{babel}

\usepackage{amssymb,amsmath,amsthm,amsfonts}
\usepackage{calc}
\usepackage{graphicx}
\usepackage{subfigure}
\usepackage{gensymb}
\usepackage{natbib}
\usepackage{url}
\usepackage[utf8x]{inputenc}
\usepackage{amsmath}
\usepackage{graphicx}
\graphicspath{{images/}}
\usepackage{parskip}
\usepackage{fancyhdr}
\usepackage{vmargin}
\usepackage{verbatim}
\setmarginsrb{3 cm}{2.5 cm}{3 cm}{2.5 cm}{1 cm}{1.5 cm}{1 cm}{1.5 cm}

\title{Tu titulo}					% Titulo
\author{Tu nombre}					% Autor
\date{\today}						% Fecha


\makeatletter
\let\thetitle\@title
\let\theauthor\@author
\let\thedate\@date
\makeatother

\pagestyle{fancy}
\fancyhf{}
\rhead{\theauthor}
\lhead{\thetitle}
\cfoot{\thepage}

\begin{document}

\tableofcontents
\pagebreak

%%%%%%%%%%%%%%%%%%%%%%%%%%%%%%%%%%%%%%%%%%%%%%%%%%%%%%%%%%%%%%%%%%%%%%%%%%%%%%%%%%%%%%%%%

\section{Trabajo realizado en el proyecto.}

\section{Variables de entorno.}

Antes de trabajar el código, hay que agregar unas variables de entorno en el sistema.

Para ello, se hace clic en Inicio, situado en la esquina inferior izquierda de la pantalla, y teclear ‘Variables de entorno’ para buscar esta configuración. Hacer clic en ‘Editar las variables de entorno del sistema’, el resultado que aparecerá. Y en la ventana que se abrirá, hacer clic en ‘Variables de entorno’.

Por si acaso, en mi caso particular, he agregado las variables de entorno tanto para mi propio usuario como para el sistema completo. Las variables de entorno que deberán agregarse son:

Variable: Python35-32. Valor: C: \textbackslash Users \textbackslash Pc \textbackslash AppData \textbackslash Local \textbackslash Programs \textbackslash Python \textbackslash Python35-32.
Variable: Python36. Valor: C: \textbackslash Users \textbackslash Pc \textbackslash AppData \textbackslash Local \textbackslash Programs \textbackslash Python \textbackslash Python36.

(Siendo ‘Pc’ el nombre de mi usuario; aquí se debe introducir el nombre de usuario correspondiente.)

Para agregar una variable de entorno, hacer clic en ‘Nueva...’. Hay dos botones ‘Nueva...’. El que se ubica arriba se agrega en el usuario específico del sistema, y el que se ubica abajo se agrega para todo el sistema en general. En ‘Nombre de la variable’ y ‘Valor de la variable’, introducir los datos mencionados anteriormente, y hacer clic en ‘Aceptar’.

\end{document}