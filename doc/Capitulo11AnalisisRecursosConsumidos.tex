\documentclass[12pt]{article}

\usepackage[utf8x]{inputenc}
\usepackage[spanish,es-noshorthands]{babel}

\usepackage{amssymb,amsmath,amsthm,amsfonts}
\usepackage{calc}
\usepackage{graphicx}
\usepackage{subfigure}
\usepackage{gensymb}
\usepackage{natbib}
\usepackage{url}
\usepackage[utf8x]{inputenc}
\usepackage{amsmath}
\usepackage{graphicx}
\graphicspath{{images/}}
\usepackage{parskip}
\usepackage{fancyhdr}
\usepackage{vmargin}
\usepackage{verbatim}
\setmarginsrb{3 cm}{2.5 cm}{3 cm}{2.5 cm}{1 cm}{1.5 cm}{1 cm}{1.5 cm}

\title{Tu titulo}					% Titulo
\author{Tu nombre}					% Autor
\date{\today}						% Fecha


\makeatletter
\let\thetitle\@title
\let\theauthor\@author
\let\thedate\@date
\makeatother

\pagestyle{fancy}
\fancyhf{}
\rhead{\theauthor}
\lhead{\thetitle}
\cfoot{\thepage}

\begin{document}

\tableofcontents
\pagebreak

%%%%%%%%%%%%%%%%%%%%%%%%%%%%%%%%%%%%%%%%%%%%%%%%%%%%%%%%%%%%%%%%%%%%%%%%%%%%%%%%%%%%%%%%%

\section{Análisis de los recursos consumidos.}

A continuación, se mostrarán y explicarán los resultados del uso de la CPU y del consumo de la memoria RAM en el PC, en el administrador de tareas de Windows 10, durante la ejecución del software en distintos periodos de tiempo, y para un valor de área mínima de helióstato de 1000, y 50 de ancho y alto. Se ignorarán los resultados de procesos y programas ajenos a los de este software. En cada explicación, se mencionará y asignará también un instante de tiempo numérico con el fin de relacionar dichas explicaciones con más explicaciones posteriores en este apartado.



Ventana del administrador de tareas de Windows, usada para medir el uso de la CPU y de la memoria RAM en el sistema por cualquier proceso y programa en ejecución.




Instante de tiempo 1. Al iniciar la ejecución del software, cuando está entrando el primer helióstato desde el lado izquierdo en el vídeo y aún no se muestra en su totalidad, Python consume un 2,1\% de CPU y 33,7 MB de memoria RAM, mientras que la terminal de Windows consume un 0\% de CPU y 9,1 MB de memoria RAM. Los FPS son de 58,11.




Instante de tiempo 2. Cuando el primer helióstato ya ha llegado al centro del vídeo, Python consume un 4,2\% de CPU y 34,2 MB de memoria RAM, mientras que la terminal de Windows consume un 1,2\% de CPU y 9,1 MB de memoria RAM. Los FPS son de 60,30.




Instante de tiempo 3. Cuando está entrando desde la izquierda del vídeo el segundo helióstato, Python consume un 2,0\% de CPU y 34,3 MB de memoria RAM, mientras que la terminal de Windows consume un 0,5\% de CPU y 9,1 MB de memoria RAM. Los FPS son de 61,65.




Instante de tiempo 4. Cuando el segundo helióstato prácticamente se ha fusionado con el helióstato permanecido en el centro del vídeo, Python consume un 4,3\% de CPU y 34,3 MB de memoria RAM, mientras que la terminal de Windows consume un 1,6\% de CPU y 9,1 MB de memoria RAM. Los FPS son de 61,87.




Instante de tiempo 5. Cuando está entrando desde la izquierda del vídeo el tercer helióstato, Python consume un 3,5\% de CPU y 34,3 MB de memoria RAM, mientras que la terminal de Windows consume un 2,1\% de CPU y 9,0 MB de memoria RAM. Los FPS son de 61,84.




Instante de tiempo 6. Cuando el segundo helióstato prácticamente se ha fusionado con el helióstato permanecido en el centro del vídeo, Python consume un 3,4\% de CPU y 34,3 MB de memoria RAM, mientras que la terminal de Windows consume un 2,6\% de CPU y 9,0 MB de memoria RAM. Los FPS son de 61,93.




Instante de tiempo 7. Cuando está entrando desde la izquierda del vídeo el cuarto y último helióstato, Python consume un 2,2\% de CPU y 34,3 MB de memoria RAM, mientras que la terminal de Windows consume un 0,9\% de CPU y 9,0 MB de memoria RAM. Los FPS son de 62,28.




Instante de tiempo 8. Cuando el cuarto helióstato comienza a iniciar la fusión con el helióstato permanecido en el centro del vídeo, Python consume un 3,8\% de CPU y 34,3 MB de memoria RAM, mientras que la terminal de Windows consume un 2,2\% de CPU y 9,0 MB de memoria RAM. Los FPS son de 62,33.




Instante de tiempo 9. Cuando uno de los helióstatos fusionados con el helióstato del centro del vídeo trata de salir del mismo (todavía no se ha completado este procedimiento). Python consume un 1,2\% de CPU y 34,3 MB de memoria RAM, mientras que la terminal de Windows consume un 1,0\% de CPU y 9,0 MB de memoria RAM. Los FPS son de 62,65.




Instante de tiempo 10. Cuando el helióstato ya se ha separado del helióstato central del vídeo, Python consume un 1,4\% de CPU y 34,3 MB de memoria RAM, mientras que la terminal de Windows consume un 0,5\% de CPU y 9,0 MB de memoria RAM. Los FPS son de 62,68.




Instante de tiempo 11. Cuando únicamente permanece el helióstato central del vídeo (el otro helióstato ya se ha ido a la izquierda del vídeo), Python consume un 2,0\% de CPU y 34,3 MB de memoria RAM, mientras que la terminal de Windows consume un 1,3\% de CPU y 9,0 MB de memoria RAM. Los FPS son de 62,74.




Instante de tiempo 12. Cuando otro de los helióstatos fusionados con el helióstato del centro del vídeo trata de salir del mismo (todavía no se ha completado este procedimiento). Python consume un 3,1\% de CPU y 34,3 MB de memoria RAM, mientras que la terminal de Windows consume un 0,9\% de CPU y 9,0 MB de memoria RAM. Los FPS son de 62,77.




Instante de tiempo 13. Cuando los dos helióstatos están a punto de separarse entre sí. Python consume un 3,3\% de CPU y 34,3 MB de memoria RAM, mientras que la terminal de Windows consume un 0,9\% de CPU y 9,0 MB de memoria RAM. Los FPS son de 62,80.




Instante de tiempo 14. Cuando ambos helióstatos ya se han separado entre sí, Python consume un 3,3\% de CPU y 34,3 MB de memoria RAM, mientras que la terminal de Windows consume un 0,9\% de CPU y 9,0 MB de memoria RAM. Los FPS son de 62,81.




Instante de tiempo 15. Cuando el helióstato de la izquierda ya casi ni se ve porque este se ha separado bastante del helióstato del centro del vídeo, Python consume un 2,3\% de CPU y 34,3 MB de memoria RAM, mientras que la terminal de Windows consume un 0,7\% de CPU y 9,0 MB de memoria RAM. Los FPS son de 62,84.




Instante de tiempo 16. Cuando otro (el tercero ya) de los helióstatos fusionados con el helióstato del centro del vídeo trata de salir del mismo (todavía no se ha completado este procedimiento). Python consume un 1,6\% de CPU y 35,0 MB de memoria RAM, mientras que la terminal de Windows consume un 1,5\% de CPU y 9,0 MB de memoria RAM. Los FPS son de 62,93.




Instante de tiempo 17. Cuando los dos helióstatos están a punto de separarse entre sí. Python consume un 2,0\% de CPU y 35,0 MB de memoria RAM, mientras que la terminal de Windows consume un 1,3\% de CPU y 9,0 MB de memoria RAM. Los FPS son de 62,95.




Instante de tiempo 18. Cuando el helióstato de la izquierda ya casi ni se ve porque este se ha separado bastante del helióstato del centro del vídeo, Python consume un 2,8\% de CPU y 35,0 MB de memoria RAM, mientras que la terminal de Windows consume un 1,2\% de CPU y 9,0 MB de memoria RAM. Los FPS son de 62,95.




Instante de tiempo 19. Cuando el helióstato del centro del vídeo, el último de todos, empieza a trasladarse hacia la izquierda con el fin de salirse del vídeo, Python consume un 1,7\% de CPU y 35,0 MB de memoria RAM, mientras que la terminal de Windows consume un 1,8\% de CPU y 9,0 MB de memoria RAM. Los FPS son de 62,97.




Instante de tiempo 20. Cuando dicho helióstato está a punto de desaparecer del vídeo (aún se muestra parcialmente a la izquierda del mismo), Python consume un 2,7\% de CPU y 35,0 MB de memoria RAM, mientras que la terminal de Windows consume un 1,0\% de CPU y 9,0 MB de memoria RAM. Los FPS son de 63,01.




Instante de tiempo 21. Cuando finalmente no permanecen helióstatos en el vídeo (final del mismo, ya se han ido todos), Python consume un 1,0\% de CPU y 35,0 MB de memoria RAM, mientras que la terminal de Windows consume un 0,2\% de CPU y 9,0 MB de memoria RAM. Los FPS son de 63,04.

Conclusiones: durante toda la ejecución del programa, no se experimentan apenas variaciones en el procesamiento de la CPU y en el uso de la memoria RAM. Casi siempre han sido de los siguientes valores (estimados):

				CPU (\%)	Memoria RAM (MB)
Python				2,5		34,3
Terminal de Windows		1,25		9,1

Y lo mismo con los FPS (fotogramas por segundo), que solían ser de 62.

Además, la ejecución del programa se realiza rápida, en apenas 1 minuto (lo mismo que dura el vídeo), porque se han usado vectores NumPy para analizar con muchísima rapidez y eficiencia los píxeles de cada contorno o helióstato del vídeo, en lugar por ejemplo de realizar esto con dos bucles ‘for’ que tienen unos costes de procesamiento más elevados. También por ello de que el consumo de CPU y memoria RAM ha sido siempre mínimo.

\end{document}