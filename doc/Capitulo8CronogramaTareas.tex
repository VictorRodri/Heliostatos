\documentclass[12pt]{article}

\usepackage[utf8x]{inputenc}
\usepackage[spanish,es-noshorthands]{babel}

\usepackage{amssymb,amsmath,amsthm,amsfonts}
\usepackage{calc}
\usepackage{graphicx}
\usepackage{subfigure}
\usepackage{gensymb}
\usepackage{natbib}
\usepackage{url}
\usepackage[utf8x]{inputenc}
\usepackage{amsmath}
\usepackage{graphicx}
\graphicspath{{images/}}
\usepackage{parskip}
\usepackage{fancyhdr}
\usepackage{vmargin}
\usepackage{verbatim}
\setmarginsrb{3 cm}{2.5 cm}{3 cm}{2.5 cm}{1 cm}{1.5 cm}{1 cm}{1.5 cm}

\title{Tu titulo}					% Titulo
\author{Tu nombre}					% Autor
\date{\today}						% Fecha


\makeatletter
\let\thetitle\@title
\let\theauthor\@author
\let\thedate\@date
\makeatother

\pagestyle{fancy}
\fancyhf{}
\rhead{\theauthor}
\lhead{\thetitle}
\cfoot{\thepage}

\begin{document}

\tableofcontents
\pagebreak

%%%%%%%%%%%%%%%%%%%%%%%%%%%%%%%%%%%%%%%%%%%%%%%%%%%%%%%%%%%%%%%%%%%%%%%%%%%%%%%%%%%%%%%%%

\section{Cronograma de tareas.}

20 de febrero de 2.017: escogí este TFG de caracterización de helióstatos. Es un proyecto realizado a medias de Diego Zapata Hernández y que yo decidí ayudarle y finalizarlo. Conocí y hablé por primera vez con mi tutor de este TFG, Vicente González Ruiz, y el co-tutor, Luis Yebra. Vicente me explicó algunos conceptos, como la normativa de entrega y presentación del TFG, y qué código del proyecto de Diego debía yo trabajar. Vicente me dijo que yo le dijera a Luis (y así hice) una información concreta del TFG para que él nos explicase la siguiente tarea que yo debía hacer.

22 de febrero de 2.017: he ejecutado el código inicial de Diego y he agregado el código que Vicente me indicó dos días antes. Ambas cosas funcionan correctamente.

23 de febrero de 2.017: ya dispongo de un código que dado un vídeo, calcula el centroide de la proyección de cada imagen del vídeo en tiempo real. Exactamente, calcula en tiempo real el centroide de cada proyección (helióstato) de cada imagen del vídeo, y aparece un recuadro verde en cada helióstato en el vídeo. Vicente me dijo que intentara yo eliminar, de ese código, la parte de procesamiento de Diego, y yo le pregunté qué parte era esa, y dónde estaba. Aunque finalmente yo no pude hacerlo, porque no sabía qué código era (y no era) de procesamiento, y Vicente me ayudó días después en tutorías (27 de febrero), y ya lo conseguimos. Vicente me dijo que ya podía yo incluir más cálculos de helióstatos, aparte del centroide. Luis nos comentó la siguiente tarea: yo debía calcular la integral en dos dimensiones dentro del contorno calculado para cada imagen.

24 de febrero de 2.017: Vicente me dijo que el código de procesamiento (y que por tanto yo debía eliminar) era todo lo que no tenía que ver con el cálculo del contorno y el centroide, como filtrados, pirámides, umbralizaciones y correlaciones.

26 de febrero de 2.017: tuve problemas a la hora de eliminar código, ya que algunas instrucciones dependían de otras.

27 de febrero de 2.017: Vicente me dijo que eliminara de código lo que no necesitaba. Aunque después, en este mismo día, fui a tutorías y él me ayudó a eliminar todo el código de procesamiento de Diego, y dejando instrucciones de código que sí eran necesarias y que realmente no había que quitarlas, solucionando así el problema que a mí me costó lograr.

1 de marzo de 2.017: intenté adaptar el código para que funcione para todo un vídeo, en vez de para una imagen, pero no lo logré.

7 de marzo de 2.017: olvidé asistir al despacho con Vicente para que me resolviera dudas del TFG.

19 de marzo de 2.017 (aprox.): ya he adaptado el código para que el programa lea y trabaje todos los fotogramas del vídeo de helióstatos cargado.

27 de marzo de 2.017: conseguí que el programa funcionase con los cálculos que me dio Vicente. Le pregunté a Luis sobre qué más cálculos deberían hacerse.

28 de marzo de 2.017: Luis me responde a la pregunta que yo hice ayer. Hay que calcular una integral de superficie dentro de un contorno de proyección, y su resultado será una aproximación potencia total incidiendo en el blanco y reflejada por el helióstato.

17 de abril de 2.017: le agregué al código una instrucción para calcular el área del contorno del helióstato, y supuse que así completaría correctamente la tarea de Luis.

?

4 de mayo de 2.017: he hecho una primera versión del anteproyecto, y se la enseñé a Vicente por si tenía fallos antes de enviarla.

5 de mayo de 2.017: he logrado que genere el programa Python un archivo TXT con las áreas en cada fotograma del vídeo, y con el programa Gnuplot proporcionar ese archivo para generar una gráfica y guardarla como PDF.

8 de mayo de 2.017: Vicente me dijo sobre la gráfica que yo le mostré el otro día, que es raro que la gráfica indique que primero se tiene un área de helióstato grande en un fotograma determinado del vídeo de helióstatos, y que en el siguiente fotograma indique que el área es muy pequeña, y que en la realidad ambos fotogramas muestran un área de helióstato grande.

19 de junio de 2.017: el problema anterior todavía no lo pude solucionar, así que para examinarlo mejor, le agregué código que permitía que el programa pausase su ejecución automáticamente cuando lea un fotograma del vídeo de helióstatos con un área de helióstatos menor o igual que 100, y que al pulsar una tecla cualquiera en el teclado del PC, reanudara su ejecución, y así sucesivamente. También intenté que el programa mostrara en tiempo de ejecución el vídeo de los helióstatos, pero no lo logré de momento.

?

3 de julio de 2.017: le expliqué a Vicente, tal y como me dijo que yo hiciera, información que encontré en internet sobre la instrucción que localizaba contornos en helióstatos y más áreas, en el vídeo de helióstatos.

5 de julio de 2.017: Vicente me dijo que intentara pintar los contornos que el programa detectaba en el vídeo de helióstatos, para comprobar qué contornos se están detectando cuando el área de la proyección es cero.

1 de agosto de 2.017: he conseguido adaptar el programa para que guarde en el PC todas las imágenes y contornos con área de proyección tendiendo a cero. Aún así, seguí sin comprender bien por qué el programa indica que estas imágenes concretas son de área cero.

20 de septiembre de 2.017: Vicente me dijo (y también a Luis) que ese problema se puede deber a que, para cada fotograma del vídeo de helióstatos, hay más de un contorno (puntos pequeños o mini-contornos que merodean al contorno principal y grande), y que para solucionarlo, probablemente bastará con que el programa seleccione siempre el área del contorno más grande, para cada fotograma.

21 de septiembre de 2.017: le expliqué a Vicente más información sobre la instrucción que localizaba contornos en helióstatos, aunque él me dijo después que me preocupara mejor por cómo generar la figura geométrica que contiene los objetos de las escenas, y que si aparece más de una mancha, para cada fotograma del vídeo de helióstatos, que calcule el área de todas y luego el programa selecciona la más grande de todas, o que el usuario elija una de ellas.

27 de septiembre de 2.017: le recordé a Vicente de que seguía teniendo el problema de que en el programa, en imágenes (fotogramas del vídeo de helióstatos) con helióstatos grandes, indica que no hay área.

29 de septiembre de 2.017: conseguí completar una de las tareas de Vicente, que de cada fotograma del vídeo de helióstatos, que el programa seleccione siempre el helióstato más grande de todos, además de calcular el área de todos los helióstatos localizados. No obstante, le pregunté a Vicente que me gustaría poder calcular cuántos contornos hay en total para cada fotograma de ese vídeo, además de que este valor es variable. Y que además, le comenté que he mostrado por consola las salidas de todas las áreas calculadas de cada contorno para cada fotograma del vídeo, y siempre había una con un valor muy elevado (y las demás con valores muy pequeños). Esto no tenía sentido porque ocurría en el principio del vídeo, y ahí no hay áreas todavía.

30 de septiembre de 2.017: le enseñé a Vicente la salida por consola del programa, además de indicarle que el programa, durante su ejecución y al cabo de varios segundos, se detiene automáticamente por un error de fuera de rango.

1 de octubre de 2.017: solucioné por mí mismo el problema de que necesitaba saber cuántos contornos habían en total en cada fotograma del vídeo de helióstatos. Para ello, la instrucción que devolvía dicho valor era ‘for i in range(0,len(contours)):’.

2 de octubre de 2.017: he solucionado también el otro problema, el de que el programa seleccione siempre el contorno más grande de todos, para cada fotograma del vídeo de helióstatos.

5 de octubre de 2.017: me he dado cuenta de que el programa realmente no funcionaba correctamente, porque aparece el aviso por consola 'No se está detectando correctamente el contorno principal' que yo mismo puse cuando el primer contorno de todos es menor que 1000 (prácticamente inexistente).

8 de octubre de 2.017: le comenté a Vicente las partes del programa que sí estaban bien y en las que daban problemas. Concretamente, le dije: ‘He analizado bien lo que llevo hecho de mi código y de la salida por consola resultante, efectivamente, siempre me coge el contorno más grande, para cada fotograma del vídeo. Hasta ahí bien. Los dos problemas son: primero, es que ese contorno grande, no siempre aparece en la primera posición del array de contornos, durante el vídeo (no en el principio ni en el final del vídeo). Aparece por ejemplo en posiciones 2, 3, ó 4. Segundo problema: siempre me dice que hay un contorno grande, en todo el vídeo. Y eso no tiene sentido, está mal, porque en el principio y final del vídeo no hay contornos, ó éstos son pequeños, medio entrando/saliendo en/del área del vídeo. Por esto pongo la condición de que si en el primer contorno, es de área 1000 ó más, que señale por consola que sí se detecta el contorno; pero el problema ya comentado es que a veces ese contorno no está en la primera posición del array de contornos y por eso dice que no se detecta el contorno.’. Vicente me respondió que esto es normal si yo no ordenaba los contornos por tamaño, y que si yo usaba el contorno más grande, que es posible que no se tratase de la proyección de un helióstato.

13 de octubre de 2.017: he intentado poner nuevas líneas de código con el fin de que el programa reencuadre (en verde) los contornos que se van detectando a lo largo del vídeo de helióstatos, pero no ha funcionado. Se lo comenté a Vicente para que me ayudara en esto.

16 de octubre de 2.017: Vicente me dijo que me fijara bien en el código de Diego Zapata donde reencuadra los contornos localizados en dicho vídeo.

26 de octubre de 2.017: hablé con Vicente para quedar mañana día 27 en tutorías en su despacho para solucionar el problema de reencuadrar los helióstatos del vídeo.

30 de octubre de 2.017: ya solucioné el problema. Ahora el programa reencuadra únicamente los helióstatos reales (en cada fotograma del vídeo de helióstatos), y no los falsos helióstatos, a partir de un umbral determinado. El umbral es el tamaño del helióstato, no lo he puesto ni muy grande ni muy pequeño. Se lo comenté a Vicente para que lo tuviera en cuenta.

31 de octubre de 2.017: Vicente, viendo que yo le conté que ya solucioné el problema anterior, mandó un mensaje a Luis para tener todos juntos una nueva reunión y que Luis nos dijera la siguiente tarea que yo debía hacer.

6 de noviembre de 2.017: Luis recibió y leyó el mensaje de Vicente. Intentamos encontrar una buena hora para quedar los tres, pero no fue posible, hasta ahora.

7 de noviembre de 2.017: Entre los tres, decidimos quedar el viernes 10 de noviembre a las 11 de la mañana en el CIESOL. Este horario concreto se nos adecuaba muy bien.

10 de noviembre de 2.017: Luis nos comentó la siguiente y última tarea. Esta consistía en obtener las componentes RGB de todos los píxeles del contorno principal o central de cada fotograma del vídeo de helióstatos, elevarlos al cuadrado, y sumarlos. También, Vicente habló y explicó sobre el BitCoin, por cambiar de contexto. Esto no es necesario que yo lo implemente en el TFG.

29 de noviembre de 2.017: he conseguido que en un determinado píxel XY, el que yo especifique al programa, del fotograma del vídeo de helióstatos, me indicara cada componente RGB, por separado (de rojo, de verde y de azul), y también sumadas entre sí (R+G+B), pero que aún no he conseguido aplicar esto para un contorno entero, es decir, sumar todas las componentes RGB de todos los píxeles de un contorno.

18 de diciembre de 2.017: le comenté a Vicente que todavía no pude solucionar el problema del pasado 29 de noviembre. Le dije además lo siguiente: ‘Obtener la solución es complicada debido a que, si bien Slack es capaz de reencuadrar el contorno principal, el grande, y no los otros contornos o falsos contornos, Slack no detecta en qué coordenadas se ubica ese contorno, así como las coordenadas de los píxeles de ese contorno, a no ser que yo sea capaz de implementar esta función que va a ser difícil. Esto es lo que necesito para continuar. Insisto que sé medir las componentes RGB del píxel que yo le indique, pero necesito eso, que el programa me indique (no yo a él) las coordenadas XY de los píxeles del contorno principal.’. También le indiqué a Vicente que el programa reconocía todos los contornos (grande y demás), pero el contorno grande (y los pequeños) no indicaba las coordenadas exactas. Con contornos hacía yo referencia a los rectángulos que los contienen. Vicente me respondió que calcular la densidad de potencia de los rectángulos no era lo más conveniente porque habían píxeles dentro que no pertenecían al contorno, y que se podía obviar este problema. También me dijo que en realidad mi programa sí pintaba bien los rectángulos, y que yo debería tener acceso a las coordenadas (como la esquina superior izquierda e inferior derecha) de cada uno de ellos. Tras su respuesta, yo rectifiqué diciendo que como se pintaban los rectángulos, que tal vez sí tuviera acceso a las coordenadas de cada contorno, pero probablemente no lo tuviera a las coordenadas de todos y cada uno de los píxeles del contorno principal, y que eso fue lo que yo deseaba saber hacer, para continuar con el TFG. Vicente me respondió que es sencillo: bastaba con obtener (leer) los valores RGB del píxel ubicado en la esquina superior izquierda del rectángulo verde (que reencuadra el helióstato principal), para todos los fotogramas del vídeo de helióstatos, y que se hace lo mismo para el rectángulo completo, como hacer un bucle ‘for’ que recorra desde X hasta W, ó Y hasta H. Así decidí hacerlo, pero el programa únicamente leía la primera fila y la primera columna del helióstato, y no todas las filas y todas las columnas. Vicente también me dijo que hiciera un archivo ‘Léeme’ y que lo subiera a GitHub, donde está mi proyecto de TFG, para que cualquiera que ejecute el programa sepa cómo hacerlo.

25 de diciembre de 2.017: Vicente me dijo que yo he usado como índice de los bucles la misma variable que para los límites, y que deben ser diferentes.

27 de diciembre de 2.017: he hecho el cambio que Vicente me dijo que hiciera hace dos días, y aparentemente el código funciona bien. Además, he hecho y subido a GitHub el archivo ‘Léeme’. No obstante, le dije a Vicente que lo mirara por él mismo por si habían errores. También le dije que el programa tardaba horas en finalizar su ejecución porque debía leer todos los píxeles del contorno principal y para cada fotograma del vídeo de helióstatos. Finalmente, le pregunté si habría que contactar con Luis para que nos dijera cuál sería la próxima tarea a hacer en el TFG.

8 de enero de 2.018: Vicente me ha puesto algunos fallos y errores de mi código en GitHub, y que yo los debía de resolver cuanto antes. Además, a Vicente no le ha parecido que tuviera una ejecución lenta, y que me asegurara de que GitHub estuviera a la última versión disponible.

9 de enero de 2.018: le comenté a Luis que ya terminé la tarea que él mandó, pero que no sabía si estaba bien hecha, o habían algunos errores sin yo saberlo. También le comenté que mejor esperásemos a que Vicente dijera si tener los tres una nueva tutoría, o no.

3 de febrero de 2.018: le comenté a Vicente para que tuviésemos una tutoría y así enseñarle mi trabajo y resolver el problema que yo tenía desde hace tiempo en el código.

6 de febrero de 2.018: he tenido las tutorías con Vicente, y he mejorado el ‘commit’ que yo hice el pasado 27 de diciembre y lo he vuelto a subir a GitHub, el ‘commit’ de leer y obtener las componentes RGB de cada píxel del helióstato central de cada fotograma del vídeo de heliostatos. Concretamente, le agregué algunos cálculos para obtener la energía del reflejo de la proyección del helióstato, indicando en formato código que la energía era el cuadrado de las componentes RGB. Además, Vicente me enseñó y me dio en formato PDF un TFG de otro compañero, de temática distinta a mi TFG, para que yo me basase en cómo debía estructurar dicho TFG, aparte de hacer caso a las normativas de presentación, estructura y diseño de un TFG.

7 de febrero de 2.018: se me ocurrió ejecutar mi proyecto de otra forma, haciendo doble clic en mi proyecto simplemente. Apareció una ventana de consola con el fondo negro, y ahí sí se ejecuta todo el código en poco tiempo (un par de minutos), tal y como Vicente me dijo hace tiempo, que su ejecución (de Vicente) de mi código de mi TFG le duraba pocos minutos, en vez de horas, como me sucedía a mí en un principio.

9 de febrero de 2.018: Luis nos preguntó si necesitábamos su ayuda, pero yo le dije que de momento no era necesario, puesto que Vicente no dijo nada de quedar con Luis los tres para más tutorías. Luis respondió que de acuerdo, que cuando Vicente lo viese conveniente, él nos avisaría a los tres. Tal y como me dijo Vicente el otro día, yo subí mi TFG actual a Google Drive, y creé un enlace de tal forma que Vicente y Luis tuvieran acceso a dicho TFG, accediendo a este servicio en internet. Así podrán ver cómo lo llevo en cada momento, anotar fallos, sugerencias, y más.

10 de febrero de 2.018: le comenté a Vicente que, si él lo deseaba, que me sugeriese qué cosas nuevas podía yo agregar a mi informe, ya que a veces no se me ocurría nada. Además, le recomendé que revisase mi código recientemente actualizado a GitHub por si tuviese errores. Además, en el informe he explicado cosas sobre las energías renovables e información básica sobre los helióstatos.

26 de marzo de 2.018: le enseñé a Vicente cómo hice hasta el momento mi diagrama de flujo, del funcionamiento del código de este TFG.

28 de marzo de 2.018: de acuerdo al diagrama de flujo que le enseñé a Vicente, me respondió lo siguiente, además de decirme otras cosas: ‘¿Seguro que lees la secuencia de vídeo completa y luego la procesas? ¿No la vas procesando conforme la vas leyendo de disco? Espero a que tengas la versión definitiva del diagrama para preguntarte sobre las cajas, algunas no tengo muy claro qué es lo que haces dentro, como en la que dice "dispersar centroides". ¿Has terminado ya con la memoria del proyecto? ¿Has incluído todos los apartados que debe tener, incluyendo un cronograma de tareas (me parece recordar que lo exige la normativa)?’

31 de marzo de 2.018: le respondí a Vicente lo siguiente: ‘De acuerdo a tu primera pregunta: ¿te refieres justo al comienzo del diagrama? Yo así mismo lo hice en código y funciona muy bien. Después de esa parte, el diagrama de flujo (así como mi código) representa los pasos a seguir para cada fotograma del vídeo. De acuerdo a tu segunda pregunta: hay algunas cajas que no supe concretarlas muy bien, tienes razón. De acuerdo a tu tercera pregunta: todavía no he terminado el TFG. Necesito antes poder ampliarlo con muchas más páginas, como me dijiste. Lo malo es que no se me ocurren muchas más cosas (ideas) para añadir más páginas al TFG. Lo de los apartados que debe tener, pensé que era libre, porque yo en la página de la UAL de mi carrera no he encontrado un documento de TFG que indique qué apartados debe tener, así como su orden. Sólo encontré el diseño de la portada (y creo que contraportada), el diseño del anteproyecto, y poco más.’.

2 de abril de 2.018: Vicente me aclaró que la memoria del TFG tiene un contenido “libre” (evidentemente, respetando la normativa de presentación de un TFG, como la estructura, formato de fuente, presentación, márgenes, y más), y que para la evaluación de TFG el tribunal debe rellenar una rúbrica en la que se preguntan cosas como si hay un cronograma o si el alumno se explica con claridad. Además, me ha mandado un enlace web donde aparece el proyecto que él mismo describió (TFG de caracterización de proyección de helióstatos), y que el anteproyecto debía de tener al menos los apartados que ahí figuran, y que probablemente ocurrirá lo mismo para la memoria del TFG. Me solicitó que le enviase el enlace web donde indica la estructura y contenido del TFG, y así hice. Le dije además que adaptaré mi TFG a lo que ahí menciona. Además, en el informe he explicado cosas sobre la utilidad y uso de los helióstatos y del celóstato del observatorio UCM.

11 de abril de 2.018: en el informe he explicado cosas sobre el helióstato con sensor de reflexión.

15 de abril de 2.018: le hice a Vicente la siguiente pregunta: ¿cómo debo poner una cita que se aplique a múltiples párrafos, en vez de a uno solo? Además, en el informe he explicado cosas sobre la historia de la energía termosolar y del origen de la energía solar.

16 de abril de 2.018: Vicente me respondió que lo ideal en estos casos sería unir todos los párrafos en uno y al final colocar la cita.

2 de mayo de 2.018: le pregunté a Vicente que cuándo me resolvería las dudas que dejé anotadas en mi TFG, en Google Docs, como él me solía hacer.

6 de mayo de 2.018: en el informe he explicado cosas sobre el funcionamiento de la central solar.

11 de mayo de 2.018: Vicente me respondió que debido a que él andaba justo de tiempo y mis preguntas suelen ser bastantes, que por este motivo no lo hacía. Antes de hacerlo, me dijo que incorporara al TFG un índice al principio, con dos niveles: capítulo y secciones de cada capítulo. Finalmente, me preguntó si había algo en la normativa que limitase por arriba o por debajo la longitud de la memoria, y yo le respondí que no creo, que mi idea era primero finalizar el informe, y tras esto, aplicar todo el formato a mi memoria que requiere la normativa, como los márgenes superiores e inferiores, tamaños y formatos de fuente, etcétera.

15 de junio de 2.018: le avisé a Vicente de que ya hice el índice del TFG. Además, en el informe he explicado cosas sobre un proyecto que consistía en un helióstato para iluminar lugares que siempre están a la sombra, las centrales térmicas, la energía termodinámica y las centrales solares, así como generar energía eléctrica con el Sol.

16 de junio de 2.018: en el informe he explicado cosas sobre un mecanismo de seguimiento solar a partir de la tecnología espacial, la megatorre sevillana de la energía solar y más información de los helióstatos y sus usos.

17 de junio de 2.018: en el informe he explicado cosas sobre los desafíos de la astrofísica contemporánea, la energía solar, ingeniería y construcción, paneles fotovoltaicos, discos, greenmob, ángulos de los rayos del sol en la Tierra, degradación de contaminantes presentes en agua mediante fotocatálisis solar, absorción de luz en un material, y la ley de la reflexión.

18 de junio de 2.018: Vicente me dijo que revisaría y respondería en un par de días a todas mis preguntas que yo le dejé anotadas en el TFG, y si no lo hacía, que yo le mandase un mensaje. Además, en el informe he explicado cosas sobre PSA, los parques fotovoltaicos, el brillante futuro de las fábricas alimentadas con energía termosolar, PS10 y PS20, la planta de energía termosolar de concentración Gemasolar, y la central térmica Solar Power Tower.

19 de junio de 2.018: en el informe he explicado cosas sobre las centrales termosolares y la orientación de los helióstatos.

20 de junio de 2.018: Vicente ya me ha resuelto algunas de las preguntas que yo dejé anotadas en el TFG. En base a eso, yo le volví a preguntar a Vicente lo siguiente: ‘Tengo una pregunta. Donde yo puse las referencias "[1] [2]", tú me dices que no las ponga así. ¿Cómo lo pongo, así: "[1]"? Entonces, ¿qué hago con la referencia "[2]", y cómo la puedo unir con la "[1]" que van relacionadas entre sí? Y respecto al índice, lo he organizado un poco, y de momento he puesto: Introducción (con muy poco texto), Fundamentos (aquí he puesto mucho texto), Trabajo realizado en el proyecto (cantidad de texto normal), y Resultados y conclusiones (que esto aún no lo he hecho). En la sección Fundamentos, debo leer y revisar lo antes posible todo el texto. He puesto esas cosas porque creo que van muy bien con mi TFG.’. Y Vicente me respondió que está bien que exista un apartado de fundamentos, pero no puede ser que esta sección sea la mayor parte del documento. Y respecto a las referencias, que yo las puedo dejar tal y como las puse originalmente (usando “[1]”, “[2]”, etcétera), y que luego las cambiaremos fácilmente. Además, en el informe he explicado cosas sobre los sistemas de control de plantas termosolares de receptor central, como introducción a dicho informe y explicado al principio del todo.

2 de julio de 2.018: le pregunté a Vicente si ya podía dar por finalizado lo que llevaba hecho de mi TFG, ya que llevaba 114 páginas hechas, aparte de mencionarle cómo podía yo saber si toda la información que puse era correcta, y si debía agregar más cosas. Finalmente, le dije que yo debía corregir el índice, dar el formato adecuado al TFG de acuerdo a la normativa de entrega y presentación de un TFG, y el diagrama de flujo que Vicente me comentó hace tiempo que habían algunas cosas que él no comprendía bien lo que hacían. Esto último lo he mejorado y actualizado a Google Drive este mismo día, corrigiendo algunos errores que he localizado, tras una segunda revisión. Vicente me respondió que agregase un índice. Y que cuando lo haya hecho, aparte de estar bien estructurado, que le avisara.

3 de julio de 2.018: he mejorado el índice del TFG. Sin embargo, tras hacer esto, me di cuenta de que tengo muchos apartados de Introducción y pocos de Fundamentos. Y que debería organizarlo mejor, dividiendo los apartados de Introducción en otros más concretos, y creando nuevas secciones (con nuevos nombres) del índice.

4 de julio de 2.018: Vicente me dijo, tras lo que le comenté ayer día 3 de julio, que cuando yo viera la memoria perfecta, que él la revisaría para ver si estaba bien hecha.

8 de julio de 2.018: le dije a Vicente lo siguiente: ‘Buenas. Te cuento. Creo que ya he realizado un índice y memoria correctos y perfectos. Ambas cosas las he dividido en: introducción, ejemplos de helióstatos, fundamentos, futuro, trabajo realizado en el proyecto, resultados y conclusiones (bueno, en realidad, este punto todavía me falta, pero lo haré en poco tiempo), y referencias bibliográficas. Si quieres, ya le puedes echar un vistazo a mi TFG.’. En dos horas y media, Vicente me respondió que revisaría mi TFG en cuando pueda.

14 de julio de 2.018: Vicente leyó por encima mi memoria (de momento) y me dijo lo siguiente: ‘Tras una primera lectura de tu memoria te sugiero que resumas los primeros capítulos que hablan de las plataformas solares hasta un tamaño que sea comparable al capítulo en el que describes tu trabajo. Yo incluiría además, un capítulo de Python indicando las particularidades del lenguaje y por qué se ha escogido para desarrollar tu proyecto, y otro sobre OpenCV y por qué se ha escogido también. Con esto, el primer nivel del índice quedaría algo así:
1. Descripción del problema a resolver.
2. El lenguaje de programación Python.
3. La biblioteca OpenCV.
4. Propuesta desarollada.
5. Conclusiones y posibles líneas de trabajo futuro.
6. Bibliografía.
Yo continuaré revisando lo que sería el punto 4 de dicho índice.’. Le pregunté, en base a eso, que aparte de lo de Python, si lo de las plataformas solares y demás capítulos los debía resumir más, quitando mucho texto innecesario que puse, o no hacía falta.

16 de julio de 2.018: Vicente me respondió que era correcto, que el 90\% de mi memoria actual sería el punto 1 del índice que él me dijo anteayer día 14 de julio.

17 de julio de 2.018: le pregunté a Vicente lo siguiente: ‘De acuerdo. No obstante, al igual que en el TFG ejemplo que tú me pasaste (y como me dijiste), debo hacer que la descripción del problema a resolver (punto 1) y la propuesta desarrollada (descripción del trabajo, punto 4) sean prácticamente del mismo número de hojas, tengo entendido. ¿Es así? Más cosas: en el caso de la descripción del trabajo (punto 4), yo no pensaba agregar más cosas. No se me ocurrían más cosas que poner, excepto si trato de hacerlo para que tenga el mismo número de páginas que en el apartado de 'descripción del problema a resolver' (punto 1) que ahí puse muchas cosas. Y además, encontré una página de wikipedia sobre la historia de python pero por ser de wikipedia (ahí las fuentes no son siempre fiables), pues no sé si incorporarlo al tfg o no.’. Vicente me respondió lo siguiente: ‘Generalmente los 4 primeros capítulos del índice que te he propuesto tienen aproximadamente la misma extensión, cada uno. Lo que no puede ser es que uno ocupe 90 página y otro 2, siendo este último el que describe tu trabajo. A ver, cosas que debes de poner en el capítulo 4: (1) un cronograma de tareas (tuyas, no del software, desde el comienzo del proyecto), (2) descripción de los requerimientos/requisitos, (3) descripción (texto) del funcionamiento del software, (4) entradas y salidas, (5) análisis de los recursos (principalmente memoria y CPU) consumidos, (6) plataformas (más bien, sistemas operativos) que serían capaces de ejecutarlo, (7) el código en sí, COMPLETAMENTE COMENTADO, (9) ejemplos de ejecuciones y comentarios sobre los resultados de las mismas.’. Además, en el informe he explicado cosas sobre OpenCV y Python.

18 de julio de 2.018: le dije a Vicente que de acuerdo, y que empezaría a hacer el cronograma de tareas.

1 de agosto de 2.018: en el informe he explicado más cosas sobre Python. Además, le dije a Vicente por Slack de que el programa, aproximadamente en el segundo 10 de iniciar su ejecución, muestra en consola una advertencia (no un error) cuando se intenta elevar al cuadrado la R, la G y la B, del píxel X225 Y99 en el contorno (helióstato) principal del vídeo, y que no sabía por qué sucedía esto. Era un error de fuera de rango (‘overflow’). Se lo dije por si me echaba una mano con el problema.

4 de agosto de 2.018: le comenté a Vicente que he actualizado el cronograma de tareas agregando esta vez las tareas de las cosas, temas y páginas web que iba poniendo en la memoria. También le comenté que ya he puesto en principio todas las cosas que debía poner yo en el capítulo 4, aunque debería extenderme algo más. Finalmente, le recordé que él debía responderme a las dos dudas que yo le escribí por Slack los días 30 de julio y 1 de agosto.

5 de agosto de 2.018: Vicente me respondió a las dudas que yo le pregunté hace unos días. Me dijo que como sugerencia (no era obligatorio), que en la memoria, que había que poner el código comentado del programa usado y sus explicaciones de cómo funcionaba, debería estar organizado (dicho código) en funciones y comentar cada una de ellas. Es decir, comentar bloques de instrucciones. Me solicitó además de que le recordase cuál era la URL de mi repositorio de GitHub para que me viese y solucionase el problema de que el programa mostraba en consola unas advertencias (ver ‘1 de agosto’) nada más ser ejecutado. Unas horas más tarde, le proporcioné a Vicente dicha URL, actualizada desde este mismo día, y que yo debía solucionar los tres ‘issues’ que Vicente me comentó hace tiempo en GitHub (unos problemas de presentación del código que yo subí ahí), así como mejorar el ReadMe (que indicaba cómo funcionaba y se ejecutaba ese código). Más horas después, le hice una nueva pregunta: ‘Como puedes apreciar en los vídeos y consola, se está detectando el contorno principal, porque de todos los contornos detectados en el fotograma actual del vídeo, el primero de todos es el de mayor área (p. ej.: 54 34 43 21 14 24). Pero sin embargo, no aparece ni calcula en esta ocasión las sumatorias RGB al cuadrado de todos los píxeles de ese contorno principal, porque yo lo programé que hiciera esto solo cuando se detecte un contorno de ancho mayor a 70 y así ignorar falsos contornos. ¿Debería de corregir esto borrando completamente la condición esa de 'ancho mayor a 70', y hacerlo mejor para cuando se detecte el contorno principal, que haga las sumatorias RGB al cuadrado de todos los píxeles de dicho contorno principal? Si hago esto, es posible que tarde un poco en reorganizar las nuevas explicaciones y nuevo formato de código a mi memoria. Espero haberme explicado bien.’. Le adjunté también una foto (captura de pantalla de mi PC) de mi problema.

8 de agosto de 2.018: Vicente ejecutó mi código, y me dijo que mejor lo corrigiera de tal forma que el usuario pudiera introducir por parámetros y desde consola las variables deseadas, como la ruta o directorio del vídeo de helióstatos a procesar, y el ancho y alto mínimos del helióstato para ser analizado por el programa. Todo esto usando ‘argparse’ en dicho código. En lugar de que las variables queden predeterminadamente inicializadas a determinados valores en el código, y no puedan ser modificadas bajo ningún concepto por el usuario que ejecuta el programa, que así es como yo lo tenía hecho. Para finalizar, Vicente me confirmó que, respecto a las advertencias comentadas el pasado 5 de agosto, que no eran importantes pues solo sucedían al comienzo de la ejecución del programa.

9 de agosto de 2.018: respecto a la última pregunta de Vicente de ayer 8 de agosto, le respondí que no, y que si era importante hacerlo así, como él dice (con ‘argparse’).

10 de agosto de 2.018: Traté de ejecutar el programa con muchas y distintas variables de entrada (ancho y alto del helióstato), pero el problema del programa no se solucionaba (el de que el programa a veces leía bien el helióstato y otras no, en ciertas partes del vídeo de helióstatos). Además, he corregido el programa para que los argumentos proporcionados desde consola por el usuario se leyesen usando ‘argparse’, porque maneja dichos argumentos bastante mejor que de mi anterior forma.

11 de agosto de 2.018: le dije a Vicente que, referido a mi código, acabo de hacerlo ya con ‘argparse’, y que se lo he subido a ‘GitHub’. También le comenté que intenté arreglar el fallo aquel de que en algunas partes concretas del vídeo de helióstatos no calcula las sumatorias RGB de dicho helióstato, y que pese a haber probado con distintos y muchos valores de variables (ancho y alto del helióstato), el problema seguía sin solucionarse. Y que supuse que el problema podría no ser de las variables, sino en alguna o algunas líneas de código del programa.

13 de agosto de 2.018: le dije a Vicente lo siguiente: ‘El fallo aún no lo he arreglado. Ya sé en qué falla pero no sé el por qué. De acuerdo a las salidas por consola, el helióstato sólo es leído aproximadamente en la primera mitad izquierda del vídeo de helióstatos, y no en el resto (centro y derecha del vídeo). Esto significa que sólo lee los helióstatos que están entrando en el vídeo desde la izquierda y acercándose al centro del mismo, y los helióstatos que salen de ahí hacia la izquierda hasta que salen del vídeo.’. Y le mandé un dibujo ilustrativo de mi problema.

15 de agosto de 2.018: ya encontré el error al problema del programa. Es porque en los bucles 'for' X-Y que recorren todos los píxeles del contorno, yo lo puse para que analice desde la posición X de donde se ubica ahora mismo el contorno hasta su longitud (y respectivamente desde Y hasta su altura). Y creo que no es así. Debería ser desde X hasta X+longitud (y respectivamente para Y: desde Y hasta Y+altura), porque las posiciones X-Y del helióstato no son las mismas, van cambiando a lo largo del vídeo. Hice las correcciones necesarias en el código, y este problema se finalmente se solucionó. Pero acto seguido apareció un nuevo problema de ‘error: fuera de rango’, que también debía analizar y solucionar. Ocurría cuando el helióstato alcanzaba el centro del vídeo.

16 de agosto de 2.018: ya supe por qué ocurría aquel error de fuera de rango: porque estaba confundiendo las coordenadas XY del vídeo de helióstatos, y en una de esas coordenadas me estaba saliendo del vídeo, y analizaba coordenadas inexistentes en esa área del vídeo.

17 de agosto de 2.018: logré arreglar el error de ayer en el código. Ahora, el helióstato alcanza el centro del vídeo y prosigue su ejecución hasta el final. Ya parece que no hay más errores en la ejecución del programa.

18 de agosto de 2.018: he realizado algunas modificaciones en el código para que si aparecen dos helióstatos en un mismo fotograma del vídeo, que además de analizar cada helióstato por separado, que me diga el área de cada uno de ellos. También he puesto que para que se detecte y analice un helióstato, debe tener como mínimo tal área, ancho y alto, o superior. Estas tres variables se pueden pasar por parámetro en la línea de comandos.

20 de agosto de 2.018: el programa ya corregido dura unas 14 horas de ejecución, bastante tiempo pero comprensible. Un fallo mínimo es que aunque elimina todos los falsos contornos en todo el vídeo, no los elimina en el final del mismo, cuando ya han desaparecido (se han ido) todos los contornos. Ahí, no hay contornos en el vídeo normal, pero en esa parte del vídeo hay un contorno central grande al mirar el vídeo umbralizado, y por eso lo analiza inútilmente. Y otro fallo mínimo es que cuando se detecta un helióstato y lo procesa el programa, no es reencuadrado en el vídeo hasta el siguiente fotograma. Debería reencuadrarlo justo en el momento de la detección. Estos fallos debería arreglarlos.

26 de agosto de 2.018: Vicente me respondió a todas mis preguntas que yo le hice por Slack desde el 17 de agosto. Concretamente me dijo: exacto, que yo podría mostrar en consola el área de dos helióstatos que aparecen en el mismo fotograma del vídeo, aunque cree que no tiene mucho sentido; además, es mejor analizar un helióstato en ancho y alto que por valor de área; que el tiempo de ejecución del programa de 14 horas es elevado, y que lo debía reducir; y que daba igual que se analizase erróneamente por el programa un helióstato falso al final del vídeo de helióstatos, que ese helióstato no debería existir.

27 de agosto de 2.018: le pregunté por Slack a Luis sobre unas dudas que tuve en la realización del software del TFG. Las preguntas que le hice fueron: si en el vídeo de helióstatos, en un mismo fotograma aparecen dos helióstatos, ¿analizo los dos por separado, o únicamente el de mayor área? ¿O da igual como lo haga? Y lo mismo con mostrar por consola sus áreas. ¿Muestro los dos, o sólo el de mayor área? Indicar que yo ya lo tenía implementado (y explicado también en la memoria) de la segunda forma: analizar los dos helióstatos y mostrar las informaciones y áreas de ambos.

28 de agosto de 2.018: Luis me respondió que se analicen los dos helióstatos por separado, es decir, tal y como yo lo tenía hecho finalmente. Calcular el área de los helióstatos, como sus estimaciones de potencia (aproximada, inicialmente, en base a la sumatoria del valor de los componentes RGB de cada píxel del helióstato). Luis me habló de nuevo y me dijo que a ver qué le parecía a Vicente, y que si es posible, que ellos (Luis y Vicente) pudieran ver mi proyecto funcionando. Le respondí a Luis que enseñarle mi proyecto funcionando a distancia (en lugar de presencialmente) iba a ser complicado, además de que dicho proyecto dura 14 horas de ejecución para grabarlo en un vídeo y subirlo posteriormente a una nube para que él y Vicente lo viesen.

29 de agosto de 2.018: de acuerdo a lo último que le dije ayer a Vicente, me respondió: ‘No sabía que necesitase tantos recursos para ejecutarse. Con que haya unos cuantos fotogramas (3 o 4) con las proyecciones de los dos helióstatos, en la que aparezcan los contornos detectados y la potencia estimada creo que sería suficiente. ¿Sería posible obtener estos 3 o 4 fotogramas? Por otra parte, también estaría bien tener una medida del tiempo de cómputo necesario por fotograma. ¿Cómo lo ves?’. Le respondí que en mi informe que subí a la nube lo podía consultar, todo lo que él me había preguntado. Y respecto al tiempo de cómputo necesario por fotograma, eso no lo tenía hecho, pero lo implementaré.

30 de agosto de 2.018: Vicente dijo que opinaba lo mismo que Luis: ‘Analizar en paralelo todas las proyecciones de un área mínima que aparezcan concurrentemente en el vídeo. Sí así está hecho, perfecto. Y que respecto de esas 14 horas, como a Luis a mí me parece que es demasiado tiempo para los cálculos que haces. ¿Podrías recordarnos qué operaciones aplicas a cada fotograma del vídeo? También sería útil que hicieras un profiling de la aplicación y nos cuantificaras el tiempo de cada una de esas operaciones.’ Yo les respondí (a Vicente y a Luis): ‘El programa analiza y muestra por consola y en este orden los siguientes datos para cada contorno (helióstato): su valor de área (sigo sin saber sus unidades; el área solo se muestra al comienzo del análisis de un nuevo helióstato), píxel XY en análisis de ese helióstato, su ancho y alto, coordenadas de la esquina superior izquierda (y luego superior derecha e inferior derecha) del helióstato, valores de cada componente RGB de ese píxel en análisis, valores de cada componente RGB anterior elevados al cuadrado, sumatoria acumulativa de todos los valores RGB al cuadrado (componentes por separado) de todos los píxeles del contorno, y tras analizar todos los píxeles del contorno, 'unificar' las anteriores componentes R+G+B para sumarlas entre sí.’ También les dije que haría lo de cronometrar las operaciones concretas que yo hago en el programa para ver cuánto tardan, así como el tiempo de cómputo necesario por fotograma del vídeo. Finalmente, Vicente me sugirió que probara a redirigir la salida del programa a un archivo para ver si se aceleraba la ejecución. Lo hice, pero no se apreciaban diferencias en la velocidad de ejecución del programa, antes y después.

3 de septiembre de 2.018: hemos tenido una conferencia online Vicente, Luis y yo, con el fin de enseñarles mi programa en ejecución para que me dijeran posibles fallos y mejoras.

4 de septiembre de 2.018: Vicente me sugirió que optimizase más mi programa, ya que había una sección de código que consumía bastantes recursos del PC. Y que para ello, que usase la biblioteca de Python 'NumPy'. Me llevó hasta el día 9 desarrollar correctamente esto.

10 de septiembre de 2.018: seguí actualizando la memoria, ya que de nuevo tenía contenido explicado para una versión antigua del programa que tengo hecho actualmente.

25 de septiembre de 2.018: le pregunté presencialmente algunas dudas a Vicente sobre si algunas secciones de mi memoria estaban bien o no hechas.

1 de octubre de 2.018: marqué en color rojo todo aquel texto de mi memoria que es totalmente prescindible o resumible, para tenerlo en cuenta.

3 de octubre de 2.018: Luis miró por encima mi informe, y me sugirió que borrase todo el contenido que yo expliqué y que no tuviera nada que ver con helióstatos y centrales de torre (centrales de receptor central).

5 de octubre de 2.018: Vicente descubrió que mi programa no funciona siempre, para cualquier vídeo de helióstatos que se le proporcione. Porque algunos vídeos de helióstatos sí los procesaba bien el software, pero con otros no. Y me sugirió que solucionase este problema. También, Vicente y Luis me recomendaron que pasase mi memoria a LaTeX.

8 de octubre de 2.018: en cierto modo, he logrado que el programa mostrase la información de los helióstatos por consola a dos columnas, tal y como Vicente me dijo que hiciese. Aunque se muestran comas y corchetes en dicha información, y que no fui capaz de eliminar. También, he reducido la longitud de la memoria a 65 páginas el capítulo 1 de helióstatos y centrales de torre, resumiendo dicho contenido.

11 de octubre de 2.018: he reducido de nuevo la longitud de la memoria, en esta ocasión a 32 páginas, el capítulo 1 de helióstatos y centrales de torre. El objetivo es llegar a 10 páginas en este capítulo.

13 de octubre de 2.018: he reducido todavía más la longitud de la memoria a 23 páginas el capítulo 1.

15 de octubre de 2.018: Luis me sugirió que explicase también en mi informe sobre la  Ivanpah Solar Electric Generating System. Así hice.

17 de octubre de 2.018: estaba practicando con la herramienta online Overleaf con el fin de, por lo pronto, pasar el índice de mi memoria a LaTeX, y lo logré hacer.

19 de octubre de 2.018: he pasado todo mi informe a LaTeX, aunque debo corregir los errores que me producen. Al copiar únicamente texto, no hay problema, pero debo agregar el código necesario para insertar imágenes y más. También he logrado corregir el fallo que me comentó Vicente el pasado día 5 de octubre en el que el programa no siempre funcionaba para cualquier vídeo que se le proporcionase.

20 de octubre de 2.018: 

\end{document}