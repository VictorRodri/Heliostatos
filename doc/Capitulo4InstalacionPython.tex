\documentclass[12pt]{article}

\usepackage[utf8x]{inputenc}
\usepackage[spanish,es-noshorthands]{babel}

\usepackage{amssymb,amsmath,amsthm,amsfonts}
\usepackage{calc}
\usepackage{graphicx}
\usepackage{subfigure}
\usepackage{gensymb}
\usepackage{natbib}
\usepackage{url}
\usepackage[utf8x]{inputenc}
\usepackage{amsmath}
\usepackage{graphicx}
\graphicspath{{images/}}
\usepackage{parskip}
\usepackage{fancyhdr}
\usepackage{vmargin}
\usepackage{verbatim}
\setmarginsrb{3 cm}{2.5 cm}{3 cm}{2.5 cm}{1 cm}{1.5 cm}{1 cm}{1.5 cm}

\title{Tu titulo}					% Titulo
\author{Tu nombre}					% Autor
\date{\today}						% Fecha


\makeatletter
\let\thetitle\@title
\let\theauthor\@author
\let\thedate\@date
\makeatother

\pagestyle{fancy}
\fancyhf{}
\rhead{\theauthor}
\lhead{\thetitle}
\cfoot{\thepage}

\begin{document}

\tableofcontents
\pagebreak

%%%%%%%%%%%%%%%%%%%%%%%%%%%%%%%%%%%%%%%%%%%%%%%%%%%%%%%%%%%%%%%%%%%%%%%%%%%%%%%%%%%%%%%%%

\section{Instalación del software Python.}

Para este trabajo de fin de grado, se necesitaba desarrollar un software usando un programa de ordenador: Python. La última versión disponible a la hora de realizar dicho trabajo era la 3.6.4. Se ha utilizado además la versión de 64 bits, debido a que mi PC donde desarrollé el programa monta esta arquitectura de CPU.

A continuación, se explicará cómo descargar Python 3.6.4.

Abrir un explorador de internet como Google Chrome, y buscar en Google ‘Instalar Python 3.6.4 64 bits’. Pinchar en el resultado ‘Python Release Python 3.6.4 | Python.org’.



Una vez se haya accedido a esa página oficial de Python, realizar scroll hacia abajo para acceder a los archivos disponibles de esa versión de Python a descargar en el PC.



Pinchar en ‘Windows x86-64 executable installer’ para descargar en el PC el instalador de Python, el cual se encargará de instalar este programa.



La descarga se completará en cinco segundos. Pinchar en el archivo ‘EXE’ mostrado en la esquina inferior izquierda para abrir el instalador de Python.



Aparecerá esta ventana de instalación. A no ser que se desee agregar manualmente las variables de entorno de Python (con el fin de poder ejecutar Python desde la consola de comandos de Windows), hacer clic en ‘Add Python 3.6 to PATH’. El propio instalador llevará a cabo este procedimiento. A continuación, hacer clic en ‘Install Now’.



Se instalará Python en el sistema. Hay que esperar unos minutos para que este procedimiento se complete (cuando la barra verde se llene por completo).



Cuando finalice correctamente la instalación, se mostrará en la ventana del instalador ‘Setup was successful’. Hacer clic en ‘Close’ para cerrar el instalador.



El icono de Python debería aparecer en el escritorio. Hacer doble clic para abrir el programa.



Si no aparece el icono de Python en el escritorio, hacer clic en el logotipo (o bandera) de Windows mostrado en la esquina inferior izquierda de la pantalla, y teclear ‘idle’. Debería ya aparecer en el buscador de programas ‘IDLE (Python 3.6 64-bit)’. Hacer clic ahí para abrirlo.



En el caso de querer agregar el icono del programa al escritorio (si no lo estaba), desde esa misma pantalla del buscador de programas, situar el cursor sobre el programa, hacer clic con el botón derecho del ratón, y seleccionar ‘Abrir ubicación de archivo’. Se mostrará la ubicación exacta del archivo en el sistema, abriéndose la ventana del explorador de archivos de Windows.



Desde esa ventana, y haciendo clic con el botón derecho del ratón sobre el programa (acceso directo) ‘IDLE (Python 3.6 64-bit)’, seleccionar desde el menú contextual ‘Crear acceso directo’. Se creará un nuevo acceso directo a ese programa, en esta misma ventana, con nombre ‘IDLE (Python 3.6 64-bit) (2)’.



Reducir el tamaño de la ventana, de forma parecida a la mostrada en esta captura de pantalla. Tras esto, simplemente habría que arrastrar y soltar este acceso directo ‘IDLE (Python 3.6 64-bit) (2)’ desde esa ventana al escritorio. Así, ya se tiene el programa en el escritorio.

\end{document}