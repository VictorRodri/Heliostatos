\chapter{Conclusiones y posibles líneas de trabajo futuro}
 
El poder tener una estimación (lo más realista posible) de la potencia reflejada, así como identificar su contorno y centroide de un helióstato en tiempo real permitirá realizar ciertas tareas de control y operación en plantas termosolares de forma sistemática desde un computador. Por ejemplo: recalibración automática de desviaciones (offsets) de cada helióstato de forma periódica; detección de daños en la estructura si la estimación de potencia cambia de forma brusca (pérdida de facetas/espejos); estimación de la potencia contribuida por cada helióstato de forma que se pueda controlar con mayor precisión la potencia por el grupo de helióstatos en el que está operando en cada momento. La contribución más importante es que estas tareas las asumiría el computador de forma automática durante la operación de la planta, aunque con la debida supervisión del operador de planta.

Por otro lado, el cálculo de la "variable que estima la potencia" se realiza tiempo real con las herramientas actuales de prototipado del algoritmo (como Python), por lo que se puede decir que el algoritmo estaría preparado para ser aplicado en una planta termosolar, en este momento, para realizar los ensayos de calibración orientados a obtener la correlación.

Debido al control automático del diafragma de la cámara, genera una perturbación al obtener las gráficas de las estimaciones de potencia de los helióstatos para algunos vídeos, y los glitches/caídas se pueden atribuir a esto. Uno de los trabajos a futuro, podría consistir en controlar desde el programa el diafragma de la cámara para que no genere estos ruidos.

Finalmente, dar agradecimientos al centro PSA-CIEMAT por haber puesto a disposición la imágenes utilizadas en este TFG, y al centro mixto CIESOL por los espacios utilizados en las frecuentes reuniones con los tutores en la UAL.