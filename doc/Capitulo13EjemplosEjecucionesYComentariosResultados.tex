\documentclass[12pt]{article}

\usepackage[utf8x]{inputenc}
\usepackage[spanish,es-noshorthands]{babel}

\usepackage{amssymb,amsmath,amsthm,amsfonts}
\usepackage{calc}
\usepackage{graphicx}
\usepackage{subfigure}
\usepackage{gensymb}
\usepackage{natbib}
\usepackage{url}
\usepackage[utf8x]{inputenc}
\usepackage{amsmath}
\usepackage{graphicx}
\graphicspath{{images/}}
\usepackage{parskip}
\usepackage{fancyhdr}
\usepackage{vmargin}
\usepackage{verbatim}
\setmarginsrb{3 cm}{2.5 cm}{3 cm}{2.5 cm}{1 cm}{1.5 cm}{1 cm}{1.5 cm}

\title{Tu titulo}					% Titulo
\author{Tu nombre}					% Autor
\date{\today}						% Fecha


\makeatletter
\let\thetitle\@title
\let\theauthor\@author
\let\thedate\@date
\makeatother

\pagestyle{fancy}
\fancyhf{}
\rhead{\theauthor}
\lhead{\thetitle}
\cfoot{\thepage}

\begin{document}

\tableofcontents
\pagebreak

%%%%%%%%%%%%%%%%%%%%%%%%%%%%%%%%%%%%%%%%%%%%%%%%%%%%%%%%%%%%%%%%%%%%%%%%%%%%%%%%%%%%%%%%%

\section{Ejemplos de ejecuciones y comentarios sobre sus resultados.}

Al iniciar la ejecución del software desde la terminal de Windows, estando en el directorio que contiene dicho software a ejecutar, y usando el comando 'estimacion\_potencia.py Videos/varios\_heliostatos.mp4 50 50 127 2', ocurre lo siguiente:

Cuando un helióstato comienza a entrar en el vídeo de helióstatos, el programa todavía no lo reencuadra en dicho vídeo ni lo analiza. Esto se debe a que, en este instante de tiempo, el helióstato mostrado en el vídeo aún no es de ancho y alto 50 píxeles o más, tal y como solicitó el usuario por parámetros en la consola. De forma análoga ocurre cuando un helióstato está a punto de salir del vídeo: en ese instante apenas se muestra el helióstato en el vídeo, y el ancho o el alto mostrados no son mayores de 50 píxeles, siendo ignorado por el programa. También debe cumplirse el valor de umbral proporcionado por parámetro para el análisis del helióstato, 127. De lo contrario, si el helióstato se muestra en su totalidad en el vídeo, o por lo menos su ancho y alto en un instante determinado son mayores de 50 píxeles (y por supuesto el umbral del vídeo es al menos de 127), el programa lo reencuadra en el vídeo y lo analiza.

En el análisis de un helióstato, el programa calculará su ancho y alto, su área total (todo estos valores se miden en píxeles), la sumatoria acumulativa de los valores de cada componente BGR al cuadrado (cada componente por separado) de todo el helióstato, y esto mismo pero sumando las tres componentes resultantes entre sí. Los resultados de estos datos son mostrados por consola, para cada helióstato analizado. En la consola, dichos datos son representados textualmente así (respectivamente):

Ancho y alto WH del helióstato en píxeles.
Área del helióstato en píxeles.
Sumatorias BGR al cuadrado de todos sus píxeles.
Suma total BGR al cuadrado helióstato completo.

En el análisis de un helióstato, además, el programa reencuadrará en el vídeo y en tiempo de ejecución los helióstatos que se están analizando actualmente. Si en el fotograma actual de dicho vídeo simplemente aparece un helióstato, se reencuadrará en color verde. De lo contrario, si aparecen dos helióstatos al mismo tiempo en un mismo fotograma, el helióstato de la izquierda se reencuadrará en verde y el de la derecha en rojo. En cualquiera de los casos, y tal y como se comentó antes, el programa analizará el o los helióstatos y mostrará sus resultados en la consola. Tener en cuenta que cuando se analizan dos helióstatos en un mismo fotograma, el programa mostrará sus respectivos resultados en la consola en dos columnas. La primera para el helióstato reencuadrado en verde, y la segunda para el helióstato reencuadrado en rojo. Finalmente, indicar que cuando dos helióstatos, en un mismo fotograma, están fusionados (parcialmente o totalmente) o rozándose, serán tratados y analizados como si fueran uno solo.

Todo este procedimiento se repite para todos los demás fotogramas del vídeo de helióstatos, hasta que se hayan leído todos sus fotogramas, finalizando así la ejecución del programa.

\end{document}