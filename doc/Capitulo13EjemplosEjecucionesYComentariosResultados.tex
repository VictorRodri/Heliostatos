\documentclass[12pt]{article}

\usepackage[utf8x]{inputenc}
\usepackage[spanish,es-noshorthands]{babel}

\usepackage{amssymb,amsmath,amsthm,amsfonts}
\usepackage{calc}
\usepackage{graphicx}
\usepackage{subfigure}
\usepackage{gensymb}
\usepackage{natbib}
\usepackage{url}
\usepackage[utf8x]{inputenc}
\usepackage{amsmath}
\usepackage{graphicx}
\graphicspath{{images/}}
\usepackage{parskip}
\usepackage{fancyhdr}
\usepackage{vmargin}
\usepackage{verbatim}
\setmarginsrb{3 cm}{2.5 cm}{3 cm}{2.5 cm}{1 cm}{1.5 cm}{1 cm}{1.5 cm}

\title{Tu titulo}					% Titulo
\author{Tu nombre}					% Autor
\date{\today}						% Fecha


\makeatletter
\let\thetitle\@title
\let\theauthor\@author
\let\thedate\@date
\makeatother

\pagestyle{fancy}
\fancyhf{}
\rhead{\theauthor}
\lhead{\thetitle}
\cfoot{\thepage}

\begin{document}

\tableofcontents
\pagebreak

%%%%%%%%%%%%%%%%%%%%%%%%%%%%%%%%%%%%%%%%%%%%%%%%%%%%%%%%%%%%%%%%%%%%%%%%%%%%%%%%%%%%%%%%%

\section{Ejemplos de ejecuciones y comentarios sobre sus resultados.}

Se ha decidido en principio ejecutar el software proporcionando como parámetros “Videos/varios\_heliostatos.mp4”, “50” y “50”, siendo respectivamente el directorio y archivo de vídeo concreto del PC que se desea cargar, y el ancho y alto mínimos del helióstato para ser detectado y procesado por el programa. Se ha decidido usar estos parámetros porque no es necesario que el tamaño del helióstato sea demasiado grande para ser detectado y analizado por el software. Estos son los valores recomendados. Al iniciar su ejecución, se cargará e irá leyendo el vídeo de helióstatos especificado guardado en el sistema. En el comienzo de dicho vídeo, todavía no aparecen los helióstatos, que sería el instante en el que justo se acaba de ejecutar el programa. Su salida por consola en este momento es la siguiente:
 
Iniciando programa...

FPS: 54.78278740304231
FPS: 54.93204282709431
FPS: 55.07664810533634

Efectivamente, el programa aún no recoge la información de los helióstatos, puesto que todavía no se muestran en el vídeo, y por ello no se muestra en consola los resultados de los helióstatos analizados, salvo los FPS (fotogramas por segundo) del vídeo que siempre se muestran durante la ejecución del programa y para cada fotograma del vídeo leído. Con el fin de acelerar la ejecución del programa, así como evitar que el usuario lea en la consola información innecesaria, no se mostrará ningún aviso del tipo ‘No se detecta el helióstato’ (en caso de que esto ocurra) para cada fotograma del vídeo.

En el momento en que la ejecución del programa y visualización del vídeo sea de un medio helióstato que está entrando en dicho vídeo desde la izquierda, la salida por consola es la siguiente:

- Analizando el helióstato verde...

Ancho y alto WH del helióstato en píxeles:        140  146
Área del helióstato en píxeles:                   15943.5
Sumatorias BGR al cuadrado de todos sus píxeles:  1786999  1546178  1982788
Suma total BGR al cuadrado helióstato completo:   5315965

FPS: 57.22366308449604
 
La estructura de la salida por consola ha cambiado con respecto a la anterior porque ahora se está detectando un contorno con ancho y alto mayores a 50, y por tanto el programa lo analizará y lo reencuadrará en un rectángulo verde en el vídeo de helióstatos normal.

Ahora se muestra en dicha consola ‘Analizando el helióstato verde’, el ancho y alto en píxeles, su área también en píxeles, las sumatorias acumulativas de los valores de todos los píxeles BGR al cuadrado del helióstato (cada componente por separado), y la suma total de estas tres componentes anteriores entre sí.

Los datos de las sumatorias se han calculado como sigue:

Primero, el programa detecta cuáles son las coordenadas de las esquinas superior izquierda ‘XY’, superior derecha ‘X+W’ e inferior derecha ‘Y+H’ del helióstato. Solo conoce inicialmente las de la esquina superior izquierda. Para hallar su esquina superior derecha, la componente X de la esquina superior izquierda del helióstato se le suma su ancho W para así obtener la posición XY de la esquina superior derecha del helióstato en el vídeo. Lo mismo para ‘Y+H’: la componente Y de la esquina superior izquierda del helióstato se le suma su altura H para así obtener la posición XY de la esquina inferior derecha del helióstato en el vídeo. Para ello, se ha usado la línea de código ‘m = frame[y+2:y+h-1, x+2:x+w-1]’, que ha permitido delimitar las coordenadas de todas las esquinas del helióstato en el vector ‘m’, partiendo del fotograma actual del vídeo ‘frame’. La primera entrada de ‘frame’ representa el número de filas, y la segunda el número de columnas, del helióstato cargado en ‘m’. Por eso primero se han puesto las ‘Y’, y luego las ‘X’. Los ‘+2’ y ‘-1’ de ‘frame’ han permitido leer lo que es el helióstato excluyendo el rectángulo verde o rojo mostrado en el vídeo.

Tras esto, se calculan (pero no se muestran en consola) los valores de las componentes RGB del píxel en análisis para el helióstato en cuestión (habría que analizar todos los píxeles para cada helióstato), y acto seguido, estas componentes RGB se elevan al cuadrado. Cuando se eleva una de las componentes RGB al cuadrado, y esta sobrepasa el valor 255, su siguiente valor es 0, y así sucesivamente. Esto es debido a que las componentes RGB solo se representan con valores comprendidos entre 0 y 255.

Tras realizar la operación de elevar al cuadrado cada componente RGB, conforme se van recorriendo los píxeles del contorno y así hasta leer todos los píxeles que los contiene, se van sumando (acumulando) los valores R, G y B, por separado, para cada fotograma del vídeo de helióstatos. Cuando finalice este procedimiento, se mostrará en consola (indicado como ‘Sumatorias RGB al cuadrado de todos sus píxeles’) la sumatoria total de los valores de todos los píxeles RGB al cuadrado del helióstato.

Una vez recorridos todos los píxeles del contorno, el resultado final será mostrado en la consola como ‘Suma total RGB al cuadrado helióstato completo’. Lo que se ha hecho ha sido simplemente sumar los valores R más G más B de ‘Sumatorias RGB al cuadrado de todos sus píxeles’ con el fin de obtener el resultado total ‘Suma total RGB al cuadrado helióstato completo’ y guardarlo después en una variable. Concretamente, indica el resultado de la sumatoria de las variables R al cuadrado, G al cuadrado y B al cuadrado de todos los píxeles del contorno, y luego la suma de estos tres resultados entre sí.

Después, en el caso de que exista un segundo contorno (y separado del primero) en el mismo fotograma del vídeo, se analizará de igual forma que el primero (mostrado en consola esta vez como ‘Analizando el helióstato rojo’) y se mostrarán en consola sus respectivos resultados. En este caso, como no había un segundo contorno, sólo se ha analizado el único contorno disponible en dicho fotograma del vídeo, y mostrado en consola los resultados de solo ese contorno. Acto seguido, se pasará al siguiente fotograma del vídeo para seguir analizando los siguientes helióstatos (se repite todo el procedimiento mencionado en este apartado) hasta que el vídeo (o ejecución del programa) llegue a su fin.

A continuación, se mostrarán capturas de pantalla de las salidas por consola y de los vídeos de los helióstatos umbralizado y original en tiempo de ejecución, así como sus explicaciones oportunas, con el fin de demostrar el correcto funcionamiento del software desarrollado:




Parámetros utilizados para la ejecución del programa, explicados previamente.



 
El helióstato está entrando desde la izquierda del vídeo. Aunque no ha entrado completamente en él, el programa ya lo está detectando y reencuadrando en un rectángulo verde en el vídeo original de helióstatos. A medida que el helióstato se desplace en el vídeo, dicho rectángulo se irá moviendo al mismo tiempo que el helióstato lo hace, para seguir manteniendo el reencuadre. Su valor de área mostrada en la consola es apenas de 14418 porque solo se ha analizado medio helióstato por el momento, así como que la suma total de las componentes RGB al cuadrado de todos los píxeles del helióstato es de 4887874. Cuando el helióstato sea visible en su totalidad, estos resultados serán más elevados.




El helióstato ya se muestra totalmente, pero aún no ha llegado al centro del vídeo, sino que por el momento permanece en el lado izquierdo del mismo. El área del helióstato ha pasado a ser de 20523, y la suma total de las componentes RGB al cuadrado de todos los píxeles del helióstato ahora es de 6973033. Nótese que estos dos valores irán creciendo poco a poco conforme el helióstato, proveniente de la izquierda, alcanza el centro del vídeo. Estos valores serán mucho más correctos cuando el helióstato llega al centro del vídeo.




El helióstato ha llegado al centro del vídeo, y se sigue detectando por el programa y manteniendo el reencuadre. Como ahora ya se visualiza el helióstato en su totalidad, además de haber alcanzado el centro del vídeo, su área ha crecido hasta 22432, y la suma total de las componentes RGB al cuadrado de todos los píxeles del helióstato ahora es de 7214354. Estos resultados son mucho más correctos y fiables que los anteriores cuando simplemente se mostraba en el vídeo medio helióstato.




Después de que el primer helióstato del vídeo haya llegado al centro del mismo, en un momento determinado se asomará un segundo helióstato desde la izquierda. En este caso, como el ancho y alto mostrados en el helióstato todavía no son mayores a 50, no será analizado ni reencuadrado en el vídeo. Simplemente se detectará y analizará el helióstato rojo del centro del vídeo, ya que este cumple que su ancho y alto son mayores de 50, además del reencuadre.




En este caso, el nuevo helióstato que se estaba asomando desde la izquierda del vídeo, ya tiene un ancho y alto mayores a 50 puesto que ahora ya se ha asomado lo suficiente como para que el programa detecte que tiene como mínimo ese tamaño de área. Por tanto, será analizado y reencuadrado en el vídeo, además de analizar y reencuadrar el helióstato central. Como se puede apreciar, en un mismo fotograma del vídeo se han analizado por separado dos helióstatos: primero el de la izquierda, con nombre ‘helióstato verde’, y después el de la derecha, con nombre ‘helióstato rojo’. El software está programado para que se detecten y analicen hasta dos helióstatos por fotograma del vídeo, si existen. Además, se ignoran automáticamente posibles falsos contornos. Aparte de esto, el helióstato verde siempre tiene un área y suma total RGB (indicado esto último en la consola como ‘Suma total BGR al cuadrado helióstato completo.’) menor que el helióstato rojo, porque además de mostrarse así en el vídeo (el helióstato verde se muestra algo más pequeño que el helióstato rojo), el helióstato verde todavía no se ha fusionado con el helióstato rojo, todo lo opuesto a este último, que recibe muchas fusiones de los helióstatos verdes y cada vez adquiere un área y suma total RGB mayor.




El helióstato de la izquierda acaba de entrar totalmente en el vídeo, y por tanto podrá ser analizado con mayor precisión. Esta captura de pantalla demuestra que el programa puede detectar y analizar dos helióstatos mostrados completamente y al mismo tiempo en el vídeo.




Un helióstato que está entrando desde la parte izquierda del vídeo está a punto de fusionarse con el helióstato posado en el centro de dicho vídeo. En este caso, el programa reencuadra en un único rectángulo a los dos helióstatos a la vez. Así que se realizarán los cálculos habituales pero de esos dos contornos agrupados como si fuera uno. Si los dos contornos están pegados entre sí pero aún no se han fusionado, ese doble contorno tratado como si fuera uno tendrán un área y suma total RGB enormes, hasta que se fusionen completamente, en cuyo caso el área y suma total RGB se reducirán a unos valores normales.




Los dos helióstatos están a punto de completar su fusión y convertirse en uno solo. Esto hará que el conjunto de helióstatos se haga cada vez más pequeño, y por tanto, se decremente tanto su área como su suma total RGB, como se indicó previamente.




Los dos helióstatos acaban de fusionarse totalmente en uno solo.

A continuación, más helióstatos vendrán desde la izquierda del vídeo, se aproximarán y se fusionarán poco a poco con el primer helióstato que llegó y permaneció en el centro de dicho vídeo, o dicho de otra manera, en el centro del panel fotovoltaico. De esta forma, resultará un conjunto de helióstatos superpuestos entre sí, en esa ubicación. Cuando lleguen y se superpongan cuatro helióstatos entre sí, a continuación ocurrirá lo contrario: se irán separando uno por uno todos los helióstatos superpuestos, e irán desplazándose y alejándose poco a poco del panel fotovoltaico hacia la izquierda (de donde provenían), hasta que desaparezcan los cuatro por completo. Estas separaciones se mostrarán en las siguientes capturas de pantalla:



Se aprecia cómo un helióstato pequeño de la izquierda se va separando poco a poco del helióstato de su derecha o principal. Después, ese helióstato pequeño se alejará progresivamente del vídeo hacia la izquierda. Durante la separación de estos helióstatos, y al igual que cuando comenzaban a fusionarse entre sí, el área y la suma total RGB de ese conjunto de helióstatos se irá incrementando poco a poco. Y como estos helióstatos están muy pegados entre sí, el programa los analiza como uno solo.




Incluso en la separación de helióstatos, el programa realiza correctamente los mismos cálculos que cuando los helióstatos se iban fusionando entre sí. Como ahora están separados entre sí, se analizan uno por uno.




El programa sigue analizando cada helióstato (ambos reencuadrados en el vídeo) durante sus separaciones, incluyendo el helióstato de la izquierda que solo se muestra parcialmente en el vídeo (ya está saliendo del mismo) porque su ancho y alto mostrados en vídeo siguen siendo mayores a 50.




El helióstato de la izquierda está a punto de salir del vídeo, pero como su ancho y alto mostrados en el mismo ya no son de 50 o más, el programa ya no lo analiza ni lo reencuadra. Simplemente analiza y reencuadra el helióstato del centro.




Ya permanece únicamente un helióstato en el vídeo, aunque de ese helióstato todavía deben salir sus sub-helióstatos restantes que los contiene. Simplemente se analiza y se reencuadra el helióstato.

El procedimiento continúa hasta que hayan salido todos los helióstatos del helióstato central o principal del vídeo, y este último también desaparezca. Tras esto, el vídeo habrá acabado y el programa finalizará su ejecución, habiendo analizado así todos los helióstatos de cada fotograma del vídeo y mostrados en consola todos sus resultados.



Captura de pantalla del último helióstato que abandona el panel solar, junto a sus resultados en consola. Nótese que el helióstato restante va saliéndose poco a poco a la izquierda del vídeo, y en el centro del mismo ya no quedan más helióstatos.





En el final del vídeo no aparecen helióstatos, pero como hay un falso contorno con valor de ancho y alto mayores a 50 (ver vídeo umbralizado), entonces es leído y analizado por el programa de forma errónea.

Para finalizar, se mostrarán ejemplos de ejecuciones modificando a otros valores de los parámetros ancho y alto del helióstato, por ejemplo, a 180 y 130, respectivamente.




Nuevos parámetros utilizados para la ejecución del programa. Ahora, el ancho y alto del helióstato debe ser más grande que antes para ser detectado y analizado. Los resultados mostrados a continuación son de que los helióstatos solo son analizados cuando están únicamente en el centro del vídeo (estén totalmente fusionados entre sí, o sin ninguna fusión), o los dos helióstatos están rozándose entre sí (fundiéndose o separándose). Los helióstatos ubicados a la izquierda del vídeo, se muestren o no totalmente, ya no son analizados, salvo cuando el primer y último helióstato entren/salgan del mismo.



Un helióstato aparece y se ubica a la izquierda del vídeo, pero aún no está siendo detectado ni analizado por el programa porque su ancho y alto aún no son de 180 y 130 (respectivamente) o más.




El helióstato se ha desplazado un poco más hacia la derecha del vídeo, aunque aún permanece considerablemente a la izquierda del mismo, y ya sí es detectado y analizado por el programa, porque ya tiene un ancho y alto superiores a 180 y 130. Recordar que los helióstatos que están en el centro del vídeo (o casi) son de dimensiones más grandes que cuando están entrando/saliendo a/de él (que están en ese momento a la izquierda de dicho vídeo). Por eso antes no se analizaba el helióstato y ahora sí.




Mismo caso que antes, pero ahora el helióstato sí ha llegado al centro del vídeo y aún se sigue detectando por el programa por no estar muy a la izquierda.




Cuando aparecen dos helióstatos en un mismo fotograma del vídeo pero estos aún no se están tocando ni fusionando entre sí, ninguno alcanza el ancho y alto de 180 y 130 y no son analizados. Además, el helióstato del centro del vídeo reduce ligeramente sus dimensiones cuando se le aproxima desde la izquierda otro helióstato. Por eso ni se detecta el helióstato de la derecha al contrario que ocurría justo antes.



Si los dos helióstatos se tocan entre sí pero aún no se han fusionado, el conjunto ‘doble-helióstato’ alcanza un ancho y alto superiores a 180 y 130, y entonces es analizado dicho conjunto.




Los dos helióstatos completamente fusionados entre sí siguen superando el valor de ancho y alto de 180 y 130. El programa analiza este conjunto de helióstatos como si fuera uno solo.

Este procedimiento continúa completamente igual durante la incorporación y separación de los demás helióstatos al/del helióstato central o principal del vídeo.




Nuevos parámetros utilizados para la ejecución del programa. Ahora, el ancho y alto del helióstato debe ser todavía más grande que antes para ser detectado y analizado. Los resultados mostrados a continuación son de que los helióstatos únicamente son analizados cuando se ubican en el centro del vídeo (y también a la izquierda del vídeo pero solo al comienzo y final del mismo, para el primer y último helióstato, como en el caso anterior), y no en otros casos, ni siquiera cuando están rozándose entre sí (fundiéndose o separándose), como ocurría en el caso anterior.




Comienza a entrar desde el lado izquierdo del vídeo un helióstato, pero de momento no es analizado hasta que se desplace un poco más hacia la derecha.




El helióstato se ha desplazado un pelín más hacia la derecha del vídeo (pero aún permanece considerablemente al lado izquierdo del mismo), y ya sí está siendo detectado por el programa. Solo en este presente caso (primer helióstato del vídeo), y para cuando el último helióstato tienda a irse hacia la izquierda del vídeo, son detectados los helióstatos cuando se ubican a la izquierda del vídeo.




El helióstato ha alcanzado el centro del vídeo y aún es detectado por el programa. Siempre se detecta y analiza el helióstato cuando se ubica en el centro de dicho vídeo.





Cuando se muestren dos helióstatos en un mismo fotograma del vídeo (y que no están completamente fusionados entre sí), e independientemente de si se están rozando o no y de si el helióstato de la izquierda se muestra parcialmente o totalmente, el programa no los analizará.




Los dos helióstatos ya se han fusionado completamente, y como ya cumplen su ancho y alto mínimos, se analizarán en conjunto por el programa, como si fuera uno solo.

Este procedimiento continúa completamente igual durante la incorporación y separación de los demás helióstatos al/del helióstato central o principal del vídeo.

\end{document}