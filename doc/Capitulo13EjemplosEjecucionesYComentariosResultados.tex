\documentclass[12pt]{article}

\usepackage[utf8x]{inputenc}
\usepackage[spanish,es-noshorthands]{babel}

\usepackage{amssymb,amsmath,amsthm,amsfonts}
\usepackage{calc}
\usepackage{graphicx}
\usepackage{subfigure}
\usepackage{gensymb}
\usepackage{natbib}
\usepackage{url}
\usepackage[utf8x]{inputenc}
\usepackage{amsmath}
\usepackage{graphicx}
\graphicspath{{images/}}
\usepackage{parskip}
\usepackage{fancyhdr}
\usepackage{vmargin}
\usepackage{verbatim}
\setmarginsrb{3 cm}{2.5 cm}{3 cm}{2.5 cm}{1 cm}{1.5 cm}{1 cm}{1.5 cm}

\title{Tu titulo}					% Titulo
\author{Tu nombre}					% Autor
\date{\today}						% Fecha


\makeatletter
\let\thetitle\@title
\let\theauthor\@author
\let\thedate\@date
\makeatother

\pagestyle{fancy}
\fancyhf{}
\rhead{\theauthor}
\lhead{\thetitle}
\cfoot{\thepage}

\begin{document}

\tableofcontents
\pagebreak

%%%%%%%%%%%%%%%%%%%%%%%%%%%%%%%%%%%%%%%%%%%%%%%%%%%%%%%%%%%%%%%%%%%%%%%%%%%%%%%%%%%%%%%%%

\section{Ejemplos de ejecuciones y comentarios sobre sus resultados.}

Al iniciar la ejecución del software desde la terminal de Windows, estando en el directorio que contiene dicho software a ejecutar, y usando el comando 'estimacion\_potencia.py Videos/varios\_heliostatos.mp4 50 50 127 2', ocurre lo siguiente:

Cuando un helióstato comienza a entrar en el vídeo de helióstatos, el programa todavía no lo reencuadra en dicho vídeo ni lo analiza. Esto se debe a que, en este instante de tiempo, el helióstato mostrado en el vídeo aún no es de ancho y alto 50 píxeles o más, tal y como solicitó el usuario por parámetros en la consola. De forma análoga ocurre cuando un helióstato está a punto de salir del vídeo: en ese instante apenas se muestra el helióstato en el vídeo, y el ancho o el alto mostrados no son mayores de 50 píxeles, siendo ignorado por el programa. También debe cumplirse el valor de umbral proporcionado por parámetro para el análisis del helióstato, 127. De lo contrario, si el helióstato se muestra en su totalidad en el vídeo, o por lo menos su ancho y alto en un instante determinado son mayores de 50 píxeles (y por supuesto el umbral del vídeo es al menos de 127), el programa lo reencuadra en el vídeo y lo analiza.

Por parámetro en la consola, se ha indicado también que se desea leer hasta un máximo de dos helióstatos por fotograma del vídeo, porque en este vídeo pueden mostrarse hasta dos helióstatos en un mismo fotograma. Si en vez de '2' se introduce '1', cuando aparezcan dos helióstatos en un mismo fotograma, simplemente se reencuadrará en el vídeo y procesará el helióstato situado a la izquierda, y no el de la derecha.

En el análisis de un helióstato, el programa calculará su ancho y alto, su área total (todo estos valores se miden en píxeles), la sumatoria acumulativa de los valores de cada componente BGR al cuadrado (cada componente por separado) de todo el helióstato, y esto mismo pero sumando las tres componentes resultantes entre sí. Los resultados de estos datos son mostrados por consola, para cada helióstato analizado. En la consola, dichos datos son representados textualmente así (respectivamente):

Ancho y alto WH del helióstato en píxeles.
Área del helióstato en píxeles.
Sumatorias BGR al cuadrado de todos sus píxeles.
Suma total BGR al cuadrado helióstato completo.

En el análisis de un helióstato, además, el programa reencuadrará en el vídeo y en tiempo de ejecución los helióstatos que se están analizando actualmente. Si en el fotograma actual de dicho vídeo simplemente aparece un helióstato, se reencuadrará en color verde. De lo contrario, si aparecen dos helióstatos al mismo tiempo en un mismo fotograma, el helióstato de la izquierda se reencuadrará en verde y el de la derecha en rojo. En cualquiera de los casos, y tal y como se comentó antes, el programa analizará el o los helióstatos y mostrará sus resultados en la consola. Tener en cuenta que cuando se analizan dos helióstatos en un mismo fotograma, el programa mostrará sus respectivos resultados en la consola en dos columnas. La primera para el helióstato reencuadrado en verde, y la segunda para el helióstato reencuadrado en rojo. Finalmente, indicar que cuando dos helióstatos, en un mismo fotograma, están fusionados (parcialmente o totalmente) o rozándose, serán tratados y analizados como si fueran uno solo.

La tasa de fotogramas por segundo (FPS) es alta, de aproximadamente 60 FPS, porque se está procesando un vídeo de dimensiones reducidas, y gracias también al uso de matrices vectorizadas (biblioteca 'NumPy' de Python) para la carga, procesamiento y guardado de todos los datos de todos los píxeles de cada helióstato.

Todo este procedimiento se repite para todos los demás fotogramas del vídeo de helióstatos, hasta que se hayan leído todos sus fotogramas, finalizando así la ejecución del programa.

El contenido de este vídeo son de cuatro helióstatos que entran poco a poco en el vídeo desde el lado izquierdo, y se van ubicando en el centro. Los demás (y siguientes) helióstatos que también llegan desde la izquierda se solapan con el helióstato o helióstatos permanecidos en el centro de dicho vídeo. Cuando los cuatro helióstatos permanezcan solapados en el centro del vídeo, se irán separando, uno por uno, los helióstatos hacia la izquierda, saliéndose cada uno del vídeo, hasta que no queden ninguno.



Al iniciar la ejecución del software desde la terminal de Windows, estando en el directorio que contiene dicho software a ejecutar, y usando esta vez el comando 'estimacion_potencia.py Videos/heliostato.MOV 50 50 200 1', ocurre lo siguiente:

Para que el helióstato sea en esta ocasión reencuadrado en el vídeo y analizado, el nuevo umbral escogido es de 200, porque el nuevo vídeo tiene una tonalidad de color más clara. En el anterior caso, bastaba con un umbral de 127 porque aquel vídeo era más oscuro. El ancho y alto escogidos del helióstato, al igual que en el caso anterior, es de 50 por 50 píxeles.

Como en este vídeo aparece como máximo un único helióstato por fotograma, se indica por parámetro en la consola que se desea analizar esta vez 1 helióstato por fotograma (en vez de 2, como sucedía en el anterior caso). Si se escribe un número mayor a 1, provocaría en el programa un error de fuera de rango.

Al ser un vídeo con unas dimensiones muy elevadas, a comparación del anterior vídeo usado en aquel caso, la tasa de fotogramas por segundo (FPS) se reduce considerablemente, y oscila en 20 FPS. Esto es debido a que el programa le llevará más tiempo analizar los helióstatos y sus píxeles, ya que dichos helióstatos tendrán unas dimensiones más elevadas, y por tanto, más píxeles a analizar dentro del helióstato.

Como en el anterior caso, el programa reencuadra el helióstato en el vídeo y lo analiza, siempre y cuando se cumplan los requisitos deseados por el usuario por parámetros en la consola, es decir, de qué tamaño y umbral debe ser el helióstato. Y también se mostrarán los resultados por consola de los helióstatos analizados por el programa. Aparece la misma información de los resultados de los helióstatos que en el anterior caso.

Todo este procedimiento se repite para todos los demás fotogramas del vídeo de helióstatos, hasta que se hayan leído todos sus fotogramas, finalizando así la ejecución del programa.

El contenido de este vídeo es de tres helióstatos que entran poco a poco en el vídeo (desde la izquierda), permanecen en el centro durante unos segundos, y se salen poco a poco (hacia la izquierda) del vídeo. Los tres helióstatos hacen esto, de uno en uno. No aparecen dos helióstatos en un mismo fotograma del vídeo como en el anterior caso, así que por tanto no habrán fusiones ni rozamientos de dos helióstatos distintos.

\end{document}