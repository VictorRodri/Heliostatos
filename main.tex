\documentclass[12pt]{article}

\usepackage[utf8x]{inputenc}
\usepackage[spanish]{babel}

\usepackage{amssymb,amsmath,amsthm,amsfonts}
\usepackage{calc}
\usepackage{graphicx}
\usepackage{subfigure}
\usepackage{gensymb}
\usepackage{natbib}
\usepackage{url}
\usepackage[utf8x]{inputenc}
\usepackage{amsmath}
\usepackage{graphicx}
\graphicspath{{images/}}
\usepackage{parskip}
\usepackage{fancyhdr}
\usepackage{vmargin}
\setmarginsrb{3 cm}{2.5 cm}{3 cm}{2.5 cm}{1 cm}{1.5 cm}{1 cm}{1.5 cm}

\title{Tu titulo}					% Titulo
\author{Tu nombre}					% Autor
\date{\today}						% Fecha


\makeatletter
\let\thetitle\@title
\let\theauthor\@author
\let\thedate\@date
\makeatother

\pagestyle{fancy}
\fancyhf{}
\rhead{\theauthor}
\lhead{\thetitle}
\cfoot{\thepage}

\begin{document}

%%%%%%%%%%%%%%%%%%%%%%%%%%%%%%%%%%%%%%%%%%%%%%%%%%%%%%%%%%%%%%%%%%%%%%%%%%%%%%%%%%%%%%%%%

\begin{titlepage}
	\centering
    \vspace*{0.0 cm}
    \includegraphics[scale = 0.13]{Logo_Uchile_modern.png}\\[1.0 cm]	% Logo Universidad
    \textsc{\LARGE Universidad de Chile}\\[2.0 cm]	% Nombre Universidad
	\textsc{\Large CC3001-02}\\[0.5 cm]				% Codigo Curso
	\textsc{\large Algoritmos y Estructuras de Datos}\\[0.5 cm]		% Nombre Curso
	\rule{\linewidth}{0.2 mm} \\[0.4 cm]
	{ \huge \bfseries \thetitle}\\
	\rule{\linewidth}{0.2 mm} \\[1.5 cm]
	
	\begin{minipage}{0.4\textwidth}
		\begin{center} \large
			\emph{Autor:}\\
			\theauthor\linebreak
			\end{center}
	\end{minipage}\\[2 cm]
	
	{\large \thedate}\\[2 cm]
 
	\vfill
	
\end{titlepage}

%%%%%%%%%%%%%%%%%%%%%%%%%%%%%%%%%%%%%%%%%%%%%%%%%%%%%%%%%%%%%%%%%%%%%%%%%%%%%%%%%%%%%%%%%

\tableofcontents
\pagebreak

%%%%%%%%%%%%%%%%%%%%%%%%%%%%%%%%%%%%%%%%%%%%%%%%%%%%%%%%%%%%%%%%%%%%%%%%%%%%%%%%%%%%%%%%%

\section{Introducción}
\section{Nudo}
\section{blabla}
\section{más apartados}
\section{Introduction}
There is a theory which states that if ever anyone discovers exactly what the Universe is for and why it is here, it will instantly disappear and be replaced by something even more bizarre and inexplicable.
There is another theory which states that this has already happened.
\section{más apartados}
\section{Lo que sea}
There is a theory which states that if ever anyone discovers exactly what the Universe is for and why it is here, it will instantly disappear and be replaced by something even more bizarre and inexplicable.
There is another theory which states that this has already happened.
\section{más apartados}
texto balalbjlsdflfjoofsajpsd hola holajodsfapaj ajjdfojpf jiojeiojf sjoiifdojofd jifdaisjo fdsjofdajsofjsdi jafodijpaf sdjiofijdifjdojf dsajoisdfji afsfasfjo aosfdpfdjoj sdojafjidsfjeojlsd dsaojidf fsa asdf ajdos
\section{más apartados}
\section{explicacion}
\section{más apartados}
\section{otro apartado}

Definición.

Un helióstato es un conjunto de espejos que establecen una superficie grande y se mueven sobre uno o dos ejes, normalmente en montura altacimutal, lo que permite, con los movimientos apropiados, mantener el reflejo de los rayos solares que inciden sobre él, se fijen en todo momento en un punto, o diminuta superficie descomponiendo en el rayo reflejado el movimiento diurno terrestre, teniendo como objetivo seguir el movimiento del sol. Haciendo esto, los rayos que refleja el helióstato pueden ser dirigidos hacia un solo punto durante todo el día.


Utilización.

Se utilizan fundamentalmente en observaciones astronómicas para mantener fija la imagen del Sol o de un astro sobre el aparato de observación. En este caso suelen ser de pequeñas dimensiones. También se utilizan en centrales solares termoeléctricas para concentrar la energía solar sobre el receptor, y conseguir así altas temperaturas. Estos helióstatos suelen ser grandes, llegando a tener más de 120 m2.

En experimentación y pruebas de materiales a altas temperaturas, un conjunto suficientemente alto de helióstatos puede concentrar los rayos solares hasta conseguir temperaturas de más de 2000 ºC.

En centrales solares termoeléctricas fuera de servicio, como es el caso de Solar Two, se utilizan en observaciones astronómicas, para poder observar la radiación de Cherenkov gracias a la concentración conseguida empleando un gran número de grandes helióstatos.


Desarrollo de los primeros helióstatos.

Los primeros helióstatos considerados como elementos industriales se desarrollaron a los inicios de la década de los ochenta para las plantas experimentales termosolares de receptor central, con el propósito de probar la viabilidad de la energía solar térmica en los procesos de producción de electricidad a escala industrial.


Helióstato con sensor de reflexión.

[5]

El helióstato con sensor de reflexión es un helióstato perteneciente a un campo solar que refleja los haces de luz que llegan a él dotado de un mecanismo de seguimiento solar. Se trata de una invención que pertenece dentro del área de la termotecnia, al campo de la producción de energía a partir de la radiación solar.

[5]

Desde mediados del siglo XX se vienen realizando investigaciones para intentar transformar esa energía en electricidad. Es por esto que se han desarrollado placas fotovoltaicas que producen directamente electricidad cuando su superficie es convenientemente activada por la luz, y distintos tipos de colectores de calor que concentrando haces de luz sobre una tubería o sobre un receptor central, que contiene un fluido logran alcanzar temperaturas suficientes como para producir grandes cantidades de vapor de agua que genera electricidad a través de una turbina, normalmente en un ciclo de Rankine.

Dada la baja potencia específica por unidad de superficie de la radiación solar, para aprovechar esta energía de manera adecuada es necesario concentrar un gran número de haces de luz sobre un mismo punto, lo que tradicionalmente se realiza por medio de espejos orientados sobre un depósito o sobre una tubería a modo de colector. En este caso la radiación es por concentración indirecta, ya que para alcanzar su objetivo los rayos han de rebotar previamente en el espejo. [5]

Objetivo general de esta invención.

El objetivo general de esta invención es el desarrollo de un dispositivo económico en su instalación, que minimice las necesidades y gastos de mantenimiento que aproveche al máximo la radiación solar que resulte rápido y fácil de instalar en cualquier ubicación. [1] [2]
Funcionamiento de la central solar

Una central solar de tipo torre central, está formada por un campo de helióstatos 

o espejos direccionales de grandes dimensiones que reflejan la luz del sol y

concentran los haces reflejados en una caldera situada sobre una torre de gran 

altura.



En la caldera, el aporte calorífico obtenido, es mezclado con un fluido térmico, y 
a continuación, el fluido es conducido hacia un generador de vapor donde 
transfiere el calor a otro fluido, este es convertido en vapor acciona los álabes del
grupo turbina-alternador para generar energía eléctrica: El fluido es
posteriormente condensado en un aerocondensador para repetir el ciclo.

La producción de una central térmica depende de una serie de factores:

1-La cantidad de horas que este fluido esté expuesto al sol.

2-El lugar donde la fábrica esté situada.

3-La calidad de los depósitos de almacenamiento del fluido.

La energía producida, después de ser transformada, es transportada mediante líneas a la red general. [8]




Energía Solar Térmica — “TORRE CENTRAL”
 
La torre central está compuesta de un campo de helióstatos los cuales siguen automáticamente al sol.
Sus costos de instalación son superiores a los de una central térmica convencional, es por eso que no se ha comercializado.
La potencia de la torre central va de los 10 a los 200MW, pero hasta ahora su utilización ha sido solamente de investigación.
Una de las limitaciones para el desarrollo de esta Torre es la situación geográfica, ya que el buen funcionamiento de esta para obtener su máximo aprovechamiento dependen de la latitud y climatología a las que están sometidas.
Algunas de las más atractivas son gran parte de Sudamérica y Centroamérica, Europa Mediterránea, África, China.
De estas locaciones, la más importante es España, ya que por su ubicación geográfica y su clima, es la que mayor potencial tiene para el desarrollo de esta y se perfila como una gran opción para el sector del aprovechamiento de la energía solar.
Un proyecto muy importante se está llevando a cabo en Sevilla, en donde la empresa Abengoa Sola construirá una torre central, la cual ya se ocupará para la producción de energía eléctrica y ya no será un proyecto de investigación.
La configuración de las torres son de dos tipos:
Una en la que los helióstatos rodean completamente a la torre central.
En la otra, los helióstatos están colocados al norte de la torre.
BENEFICIOS DE LA TORRE CENTRAL:
Conforme se haya hecho la inversión y la instalación, ya no generará gastos mayores, el único será el del mantenimiento.
Energía eléctrica prácticamente gratis.
No requiere el uso de combustibles, lo que evita el riesgo de almacenar combustibles.
No contamina, ya que la energía solar no produce desechos, residuos ni vapores.
Menos ruidosa que una termo convencional.
INCONVENIENTES:
Se debe de instalar en lugares en donde la radiación del sol sea durante el mayor tiempo posible durante todo el día.
Su rendimiento es menor que el de otros sistemas, en comparación con el de otros sistemas. Su eficiencia es de apenas de un 16% a un 20%.
Todo su sistema mecánico es más complejo que otros sistemas.
Se necesitan acumuladores de calor para poder usar este dispositivo para cuando no haya radiación.
En cuanto al funcionamiento de la torre, es muy similar al de una termoeléctrica, ya que en la torre se coloca una caldera, y dentro de esta se coloca algún líquido (agua, aceites, etc.) y este al recibir la radiación del sol eleva su temperatura de 300 a 1500 grados centígrados y esto provoca la evaporación del líquido, de ahí pasa a un acumulador, en donde se almacena energía térmica, después pasa a la turbina.
En la turbina se genera la energía eléctrica, y de ahí pasa a un generador y un transformador y finalmente a una subestación; mientras que el vapor de agua pasa a un condensador y a un sistema de enfriamiento para posteriormente pasar por una bomba a la caldera donde el ciclo se repite.
Este proceso en un futuro podría ser utilizado para diversos procesos industriales y la climatización. [10] [11] [12]


 
Paneles Fotovoltaicos

Los paneles fotovoltaicos o módulos fotovoltaicos, también llamados a veces paneles solares, están formados por un conjunto de celdas o células fotovoltaicas que producen electricidad a partir de la luz o radiación solar que incide sobre ellos.
Los paneles fotovoltaicos consisten en una red de células conectadas como circuito en serie para aumentar la tensión de salida hasta el valor deseado (usualmente se utilizan 12V ó 24V) a la vez que se conectan varias redes como circuito paralelo para aumentar la corriente eléctrica que es capaz de proporcionar el dispositivo. La vida útil media a máximo rendimiento de los paneles fotovoltaicos se sitúa en torno a los 25-30 años, período a partir del cual la potencia entregada disminuye.
El tipo de electricidad que proporcionan los paneles fotovoltaicos es de corriente continua.
La primera generación de paneles fotovoltaicos tenían una gran superficie de cristal simple capaz de generar energía eléctrica a partir de fuentes de luz con longitudes de onda similares a las que llegan a la superficie de la Tierra provenientes del Sol. Sus células fotovoltaicas se fabrican usualmente utilizando un proceso de difusión con obleas de silicio.
Un módulo o panel fotovoltaico funciona por el efecto fotoeléctrico. Este tipo de paneles fotovoltaicos se suelen montar agrupándolos en solares o parques fotovoltaicos y su efectividad depende tanto de su orientación hacia el sol como de su inclinación con respecto a la horizontal. Para ahorrar gastos de instalación y mantenimiento, los paneles fotovoltaicos se suelen montar con orientación e inclinación fija, tratando de optimizarlos al máximo en función de la latitud.
Actualmente muchos gobiernos del mundo (Alemania, Japón, EEUU, España, Grecia, Italia, Francia y otros muchos) están subvencionando las instalaciones fotovoltaicas con un objetivo estratégico de diversificación y aumento de las posibilidades tecnológicas preparadas para generar energía eléctrica de forma masiva.
Entre otros muchos usos, los paneles fotovoltaicos se pueden destinar a las siguientes aplicaciones:
Centrales conectadas a red con subvención a la producción
Estaciones repetidoras de radio.
Electrificación de pueblos en áreas remotas
Instalaciones médicas en áreas rurales.
Corriente eléctrica para casas de campo.
Sistemas de comunicaciones de emergencia.
Sistemas de vigilancia de datos ambientales y de calidad del agua.
Faros, boyas y balizas de navegación marítima.
Bombeo para sistemas de riego
Vehículos de recreo.
Sistemas para cargar acumuladores de barcos.
Fuente de energía para naves espaciales.
Postes SOS (Teléfonos de emergencia de carretera). [27]

Ejemplos de helióstatos.

PSA.

El Horno Solar es una de las instalaciones que forma parte de la Plataforma Solar de Almería, pertenece al Centro de Investigaciones Energética, Medioambientales y Tecnológicas (CIEMAT). La instalación se compone de un helióstato plano que realiza un seguimiento solar continuo, un atenuador o persiana, un espejo parabólico concentrador y la zona de ensayos situada en el foco del concentrador.
            	La concentración y distribución de la densidad de flujo en el foco es el elemento que caracteriza a un horno Solar. Esta distribución suele tener geometría gaussiana y para su caracterización se emplea una cámara digital conectada a un procesador de imágenes.
            	El helióstato refleja los rayos solares incidentes sobre el disco parabólico, éste a su vez los refleja sobre su foco, donde se encuentra situada el área de ensayos. Un atenuador situado entre el helióstato y el concentrador regula la cantidad de luz. La mesa de ensayos se encuentra bajo el foco y permite su movimiento en todas direcciones para posicionar las probetas con precisión.
            	Los helióstatos están formados por una superficie reflectiva compuesta por múltiples facetas planas que reflejan los rayos solares horizontales y paralelos al eje óptico del concentrador y hacen seguimiento continuo del disco solar. El horno solar de la PSA consta de cuatro helióstatos dispuestos en dos niveles, cada uno de los cuales enfoca a una esquina del concentrador, de manera que se asegura la iluminación completa del concentrador durante el periodo operativo (Figura 1).

Figura 1: Helióstato
            	El disco concentrador es el elemento principal del horno solar (Figura 2). Concentra la luz que le llega reflejada del helióstato, multiplicando la energía radiante en su zona focal a 7,45 m. La superficie parabólica se consigue con el uso de facetas de curvatura esférica, distribuidas según cinco radios de curvatura distinta según su distancia al foco.
Figura 2: Concentrador Parabólico
                           	El atenuador consiste en un conjunto de lamas dispuestas horizontalmente que, mediante un movimiento giratorio sobre su eje, regulan la entrada de luz solar incidente en el concentrador (Figura 3).
            	La energía total en el foco es proporcional a la radiación que pasa a través del atenuador. Está compuesto por 30 lamas dispuestas en dos columnas de 15. En posición cerrado las lamas forman un ángulo de 55º con la horizontal y en abierto 0º.

Figura 3: Atenuador
            	La mesa de ensayo es un soporte móvil situado bajo el foco del concentrador. Tiene movimiento en tres ejes (X, Y, Z) perpendiculares entre sí, y sirve para posicionar con gran precisión en el área focal las probetas a ensayar (Figura 4). 

Figura 4: Mesa de ensayo, espejo y cámara
[26]


PS10: La megatorre sevillana de la energía solar (y cómo funciona)
 

Planta Solúcar PS10
“Hay otras, pero ella fue la primera.”
La antigua Híspalis, la ciudad de Sevilla, al suroeste de España, recibe una gran cantidad de luz solar; algo así como una media de 320 días al año, a razón de nueve horas o más cada día. En el pico máximo de un verano cualquiera, la temperatura puede elevarse hasta los 50 ºC, y el sol es capaz de brillar durante más de 15 horas diarias.
En Sanlúcar la Mayor, a 18 kilómetros de Sevilla, la empresa española Abengoa construyó una estación solar de generación de electricidad.

Campo de helióstatos y torre
Inaugurada en el año 2007, la PS10 fue la primera central solar termoeléctrica (de carácter comercial) de torre central y campo de helióstatos instalada en el mundo. Utiliza paneles de espejo para generar 11 megavatios.
Sólo PS10, PS20 y Solnovas, que se reúnen en la conocida como Planta Solúcar, operan comercialmente un total de 183 MW, produciendo energía anual equivalente a 94.000 hogares, y evitando así la emisión de más de 114.000 toneladas anuales de CO2 a la atmósfera.
La PS10 está formada por 624 helióstatos y una torre solar de 114 metros de altura. El campo de espejos o helióstatos refleja la luz solar sobre un receptor en la parte superior de la torre.

Fascinante resplandor mágico
Esta torre se encuentra en el centro de la estación PS10 Solúcar. En la parte superior, el receptor solar consiste en una serie de paneles de tubos que operan a muy alta temperatura y por los que circula agua a presión. Este receptor se calienta por efecto de la luz solar y genera vapor saturado a 257 ºC. El vapor que se produce es almacenado parcialmente en unos tanques acumuladores para ser utilizado cuando no haya suficiente producción; el resto es enviado a accionar una turbina que genera la electricidad.

Esquema de funcionamiento
Los 624 helióstatos circundantes, cada uno con una superficie de 120 metros cuadrados, producen el reflejo para que el sistema pueda convertir alrededor del 17% de la energía de la luz solar en 11 megavatios de electricidad. Como punto de comparación, la planta PS2 (que se encuentra al lado) produce 20 MW de potencia mediante su torre de 160 metros de altura sobre un campo de 1.255 helióstatos.
Por lo tanto, esta planta funciona calentando agua con luz solar y, con el vapor generado, mueve turbinas para crear electricidad.

Campo y torre PS10
[14]



PS20
La PS20 (Planta Solar 20) es ahora mismo la mayor planta comercial de torre del mundo. Está situada al lado de la PS10.
A diferencia de la PS10, la PS20 cuenta con 85 hectáreas y con 1.255 helióstatos que reflejan los rayos al receptor situado en la parte superior de la torre de 160m de altura.
La PS20 dispone con un sistema de almacenamiento de 1 hora, con un receptor más eficiente y un sistema de control y operación mejor que le permiten producir 20 MW que abastecen a 10.000 hogares anualmente.

Helióstato PS10

[29]

Gemasolar
Gemasolar es una planta de energía solar por concentración con sistema de almacenamiento térmico en sales fundidas situada en Sevilla, España. Es la primera a escala comercial en aplicar la tecnología de receptor de torre central y almacenamiento térmico en sales fundidas. La superficie de esta planta ocupa aproximadamente 158 hectáreas. Características:
Potencia eléctrica de 19,9 MW.
La producción neta esperada es de 110 GWh/año.
Capaz de suministrar energía limpia y segura que reduce en más de 50.000 toneladas al año de emisiones de CO2.
La planta está formada por:
2.650 heliostatos que forman círculos concéntricos alrededor de la torre que disponen de un mecanismo que posiciona con precisión la superficie de los espejos.
Una torre que en lo alto se encuentra el receptor de haz de luz compuesto por paneles.
2 tanques: uno de sales frías y otro de sales calientes.
Un cambiador de calor; un generador y transformador.
Su funcionamiento se ilustra en la siguiente imagen:

Tanque 1: Sales frías.
Se bombean las sales a lo largo de la torre.
En el receptor de haz de luz, las sales se calientan y bajan al tanque 2 donde se almacenan a temperaturas superiores a 500º.
Tanque 2: Sales calientes.
Cambiador de agua.
Las sales al perder calor generan vapor de agua.
El vapor de agua hace que se mueva la turbina y el generador que produce la energía.
Transformador. La energía pasa al tendido eléctrico.
La planta no solo funciona como una central solar sino también como una central térmica (cuando no hay luz) gracias a su sistema de concentración de sales y a partir de ahí también generar electricidad. También podemos decir que gracias a esto la planta asegura obtención de energía las 24 horas del días en situaciones de baja insolación, en las madrugadas y varios meses al año.
Es tal el éxito de la planta sevillana que muchos profesionales del sector buscan aplicar las energías limpias a las pequeñas urbanizaciones y así hacer posible su aplicación a la vida cotidiana. [30]



Instalación de energía solar Ivanpah

El Sistema de generación eléctrica solar Ivanpah es una planta termosolar concentrada en el desierto de Mojave . Está ubicado en la base de Clark Mountain en California, a través de la línea estatal desde Primm, Nevada . La planta tiene una capacidad bruta de 392 megavatios (MW). Se despliega 173,500 helióstatos, cada uno con dos espejos de enfoque de la energía solar en calderas situados en tres centralizados torres de energía solar. La primera unidad del sistema se conectó a la red eléctrica en septiembre de 2013 para una prueba de sincronización inicial. La instalación abrió formalmente el 13 de febrero de 2014. En 2014, fue el más grande del mundo estación de energía solar térmica.
La instalación, con un costo de 2.2 mil millones de dólares, fue desarrollada por BrightSource Energy y Bechtel.
 
Descripción
El sistema Ivanpah consta de tres plantas de energía solar térmica en 4,000 acres (1,600 ha) de terrenos públicos cerca de la frontera entre California y Nevada en el suroeste de los Estados Unidos.
Los campos de espejos de helióstatos enfocan la luz solar en los receptores ubicados en torres de energía solar centralizadas. Los receptores generan vapor para accionar turbinas de vapor especialmente adaptadas.
Para la primera planta, se ordenó el mayor grupo generador de turbina de vapor con energía solar, con una turbina de recalentamiento de caja única Siemens SST-900 de 123 MW. Siemens también suministró sistemas de instrumentación y control. Las plantas utilizan la tecnología "Luz Power Tower 550" (LPT 550) de BrightSource Energy que calienta el vapor a 550 °C directamente en los receptores. Las plantas no tienen almacenaje.
La aprobación final para el proyecto se otorgó en octubre de 2010. El 27 de octubre de 2010, el gobernador de California, Arnold Schwarzenegger, el secretario del Interior Ken Salazar y otros dignatarios se reunieron en el desierto de Mojave para abrir la tierra para la construcción.
 
Consumo de combustibles fósiles
La planta quema gas natural cada mañana para comenzar la operación. El 27 de agosto de 2014, el Estado de California aprobó a Ivanpah aumentar su consumo anual de gas natural de 328 millones de pies cúbicos de gas natural a 525 millones de pies cúbicos. En 2014, la planta quemó 867,740 millones de BTU de gas natural que emitía 46,084 toneladas métricas de dióxido de carbono. En 2015, las instalaciones mostraron mayores números de producción, con aumentos del primer trimestre del 170% en el mismo período de 2014.
En 2015, el consumo de gas natural disminuyó a 564,814 millones de BTU, mientras que la producción total de energía aumentó a 652,300 MWh.
La instalación utiliza tres calderas de tubos de agua tipo D de Rentech y tres calderas de conservación durante la noche. La Comisión de Conservación y Desarrollo de los Recursos Energéticos de California aprobó para cada uno una pila de "130 pies de alto y 60 pulgadas de diámetro" y un consumo de 242,500 pies 3 / h de combustible.
 
Impacto económico
BrightSource estimó que las instalaciones de Ivanpah proporcionarían 1,000 empleos en el pico de la construcción, 86 empleos permanentes y beneficios económicos totales de 3 mil millones de dólares.
El costo estimado de construcción para la instalación fue de 5,561.00 dólares por kW.
En noviembre de 2014, los inversionistas de la instalación solicitaron una subvención federal de 539 millones de dólares para financiar su préstamo federal.
 
Rendimiento
El rendimiento de entrega de potencia contratada de 640 GW · h / año de las Unidades 1 y 3 y 336 GW · h de la Unidad 2 se cumplió en 2017, luego de una reducción drástica de la producción en los primeros años de operación, particularmente en el inicio año de 2014.
En noviembre de 2014, Associated Press informó que la instalación producía solo "aproximadamente la mitad de su producción anual esperada". La Comisión de Energía de California emitió una declaración culpando a "nubes, estelas de aviones y clima". El rendimiento mejoró en 2015 a aproximadamente 650 GW · h.
Para 2017, debido a mejoras, la planta estaba cumpliendo con los requisitos de producción del contrato.
 
Impactos ambientales
El proyecto generó controversia debido a la decisión de construirlo en un hábitat desértico ecológicamente intacto. La instalación de Ivanpah se estimó, antes de que comenzaran las operaciones, reducir las emisiones de dióxido de carbono en más de 400,000 toneladas anuales. Fue diseñado para minimizar los impactos en el entorno natural en comparación con algunas instalaciones solares fotovoltaicas porque el uso de helióstatos no requiere tanta clasificación del terreno. La instalación fue cercada para mantener alejada a la vida silvestre terrestre. Sin embargo, las aves se enfrentaron al riesgo de colisión con los espejos de helióstatos o por la combustión del flujo solar creado por el campo de espejos.
En 2012, la Asociación de Conservación de Parques Nacionales (NPCA) emitió un informe sobre el proyecto, citando preocupaciones sobre el agua, daños a los recursos visuales e impactos en importantes especies del desierto. Para conservar el escaso agua del desierto, el LPT 550 utiliza refrigeración por aire para convertir el vapor nuevamente en agua. En comparación con el enfriamiento en húmedo convencional, esto resulta en una reducción del 90 por ciento en el uso de agua. El agua se devuelve a la caldera en un proceso cerrado.
Otro problema potencial es el efecto del reflejo del espejo en los pilotos de aviones. Además, las torres de energía tienen 'unidades receptoras' en su parte superior en las que los campos del espejo enfocan su luz reflejada. Durante la operación, estas unidades receptoras se vuelven extremadamente calientes, de modo que brillan y aparecen muy iluminadas. Debido a que están muy por encima del suelo, estas unidades receptoras brillantes son una distracción visible para las personas en muchos de los KOP (puntos de observación clave). [45]
 
 
Solar Power Tower
Las Solar Power Tower (power towers o central towers) son unas estructuras parecidas a las centrales térmicas pero que utilizan una torre de gran altura para captar los rayos del sol y transformarlos en energía eléctrica. Para ello cuentan con unos espejos móviles (sun-tracking mirrors), llamados helióstatos que hacen posible que todos los rayos incidan sobre la parte superior de la torre. Los espejos son controlados por ordenadores que tienen estudiado el movimiento del sol a lo largo del día.
Una vez los rayos han llegado a la parte superior de la torre, la energía captada se utiliza para calentar agua. Este vapor mueve una turbina que está conectada a un generador eléctrico.
Actualmente son muy conocidas las Solar Power Tower que utilizan sales fundidas en vez de agua. Las sales, compuestas por un 60% de nitratos de sodio y un 40% de nitratos de potasio transportan el calor del colector hasta el generador de vapor que luego será transformado en energía eléctrica. La utilización de sales tiene una gran ventaja ya que permite almacenar el calor para seguir produciendo energía eléctrica cuando no se goza de luz solar.
Todo este proceso se representa en la siguiente figura:

Funcionamiento Solar Power Tower [31]
 
 
Tanques de Almacenamiento térmico a través de sales fundidas para centrales termosolares
Las plantas SEGS comenzaron con una potencia de 14 MW y terminaron con una potencia de 80 MW, con una capacidad instalada total de 354 MW. Estas plantas continúan operando con éxito hasta el año 2003.
El récord con este tipo de plantas inspiró a España a continuar con sus investigaciones, inaugurando en 2009 la planta termosolar Andasol-1 en Aldeire, Granada.
El objetivo del proyecto Andasol-1 es convertir la energía solar en energía eléctrica a través de un campo solar de colectores cilindro-parabólicos, un sistema de almacenamiento térmico de 6 horas de capacidad más el 25% de seguridad a base de sales fundidas y con un ciclo de vapor de 49,9 MV de capacidad.
Cuando el sol brilla los colectores del campo solar concentran la radiación sobre los tubos absorbentes y calientan el fluido hasta una temperatura de 393°. En el fluido se encuentran sales inorgánicas como Nitrato de Sodio y Nitrato de Potasio, cuando alcanzan la mayor temperatura el fluido es transportado a un  tanque caliente. Durante la noche, el tanque caliente traspasa el fluido al tanque frío, ahí las sales calientes transfieren energía al fluido y generan el vapor.
Andasol-1 logra una eficacia anual media del 16% de conversión de radiación solar a energía eléctrica. [6]



VII.- CENTRALES TERMOSOLARES
ORIENTACIÓN DE HELIOSTATOS

VII.1.- INTRODUCCIÓN

Las centrales de potencia termosolares de alta temperatura, para la transformación de la energía solar en eléctrica, mediante un ciclo termodinámico, consisten en general, en un adecuado ordenamiento de espejos, llamados helióstatos, situados sobre un terreno, ordenados y orientados automáticamente, para que en todo momento reflejen la radiación solar directa que incide sobre ellos, en un receptor situado a gran altura sobre el terreno en el que van ubicados los espejos, de forma que toda la energía se transporte al mismo tiempo por radiación.
En el diseño de una central de energía solar para la obtención de electricidad mediante un ciclo termodinámico recorrido por vapor de agua, se pueden considerar dos partes perfectamente diferenciadas,

a) El concentrador de energía solar.
b) El receptor de energía que se comporta como caldera del ciclo termodinámico.

VII.2.- RECEPTORES

El receptor puede ir instalado en el centro del campo especular, o bien, desplazado hacia el Sur, dando lugar a los campos Norte de helióstatos; el receptor debe estar situado en el campo visual de los espejos, lo cual se cumplirá tanto mejor, cuanto más elevado se encuentre, minimizándose así los problemas de interferencia y solapamiento entre espejos vecinos.
Desde un punto de vista relativo a la absorción de energía, los receptores pueden ser de dos tipos:

a) De cavidad.
b) De recepción energética exterior.

Los de cavidad pueden ser de eje vertical o de eje horizontal, tienen una abertura por la que penetran los rayos solares reflejados, que deben tener unas dimensiones mayores que las de los espejos más alejados, teniendo presente la dispersión de la luz reflejada. Dentro de la cavidad se puede conseguir una absorción de luz de hasta un 95%.
Los receptores de cavidad de eje horizontal parecen ofrecer mayores ventajas para cuando la altura de la torre sea pequeña; en cambio, cuando los helióstatos están muy próximos a la torre, se utilizan cavidades de eje vertical.
Los receptores que no son de cavidad, y que por lo tanto absorben la energía solar por su parte exterior, se diseñan generalmente como volúmenes de revolución, pudiendo ser su eje de simetría vertical u horizontal. Los tubos absorbentes se disponen externamente, formando la superficie lateral del receptor.
Si el fluido que circula por los tubos absorbentes se vaporiza y recalienta en ellos, podrá utilizarse directamente en una turbina apropiada, acoplada convenientemente a un alternador.
El vapor condensará a la salida de la turbina y mediante un sistema de bombeo se introduce de nuevo al fluido en el receptor, cerrándose así un ciclo termodinámico.
El receptor puede diseñarse de forma que sus tubos absorbentes de energía cumplan condiciones parecidas a las que soportarían en la cámara de combustión y radiación de una central térmica.

VII.-148 VII.3.- EL CAMPO CONCENTRADOR

El campo concentrador está conformado por los helióstatos, que están formados por una serie de espejos planos, dispuestos convenientemente sobre una estructura soporte; pueden tener diversas geometrías, dependiendo fundamentalmente del tipo de receptor; en todo momento deben seguir el movimiento aparente del Sol, bien en forma individual mediante células ópticas, o en forma colectiva mediante ordenador, en el que su programa puede ser modificado diariamente, permitiendo a su vez seguir al Sol aún con cielo nublado.
La energía consumida para la dirección y orientación de los helióstatos es relativamente baja, ya que para un solo helióstato consume 60 W.
Una forma de diseño del concentrador de energía es la de corona circular, en la que los helióstatos irían dispuestos según un ordenamiento a base de anillos concéntricos; en principio se pueden suponer de forma que no dejasen entre sí ningún espacio vacío.
Según sea la posición del Sol, los helióstatos pueden interferirse mutuamente, en el sentido de que uno de ellos puede servir de pantalla de la radiación solar directa a uno o más que estén detrás, produciendo un efecto de sombra, o bien, bloquear la radiación solar reflejada por los helióstatos contiguos posteriores, reduciendo de esta forma la energía que es posible enviar sobre el receptor. [32]



Celóstato

En el siglo XIX, cuando no existían fuentes de luz artificial intensas como lámparas eléctricas o arcos voltaicos, era muy común utilizar la luz del sol en los experimentos de óptica. La forma más simple de obtener un haz de luz en un laboratorio consistía en colocar un espejo en el exterior, de manera que reflejara la luz del sol a la sala donde se realizaba el experimento. Si se dotaba al espejo de un movimiento de rotación adecuado se podía conseguir que la dirección del haz permaneciera constante durante varias horas, independientemente de la posición aparente del sol. En este caso, se tendría un helióstato.

Si se añade un segundo espejo, regulable en altura y orientación, se conseguía un celóstato. Este segundo espejo es importante porque la declinación del sol va variando a lo largo del año y por tanto la dirección del haz se proyectará a diferentes alturas. Con este segundo espejo se puede recoger la luz del primer espejo y enviarla a un instrumento astronómico, asegurando que la luz irá siempre al mismo sitio independientemente de la hora del día y de la época del año.

Un celóstato es un aparato consistente fundamentalmente en dos espejos, cuya misión es enviar la luz del sol en una dirección fija donde se encuentra un telescopio que produce las imágenes. Este telescopio está fijo y puede disponer de un espectrógrafo de alta resolución para observar el espectro de Sol.

El espejo primario y su montura tienen la misión de reflejar la luz en una dirección dada, al menos durante un día.

El espejo secundario y su montura tienen la misión de recoger el haz del primario y mandarlo en una dirección fija (durante todo el año), donde se puede colocar otro instrumento astronómico que recoja la luz. [4]



RE<C: resumen del proyecto de helióstatos

Introducción

RE<C fue una iniciativa de Google para impulsar la innovación en energía renovable, con el objetivo de hacer que la energía renovable sea lo suficientemente barata para competir cara a cara con las centrales eléctricas de carbón. Google formó un equipo de ingeniería para desarrollar tecnologías prometedoras en el campo de la generación de energía solar. Centramos entonces nuestros esfuerzos de ingeniería en concentrar la energía solar (CSP).

Las plantas de energía solar concentradas usan espejos o lentes para enfocar una gran cantidad de luz solar sobre un objetivo que absorbe calor, llamado receptor. El intercambiador de calor del receptor crea vapor a alta presión, que luego impulsa una turbina para alimentar un generador eléctrico. La refrigeración por agua en spray se utiliza normalmente para condensar el vapor. Estas plantas de energía CSP requieren que las economías de escala sean rentables, y con frecuencia tienen una potencia de 50 MW o más. El uso del agua puede ser un factor que limite la adopción de plantas de CSP a vapor, ya que el área con la mayor cantidad de luz solar que sería ideal para CSP a menudo tiene recursos hídricos limitados.

Nos enfocamos en el diseño y desarrollo de una CSP modular de "torre de energía" que utiliza un motor de turbina de gas más pequeño (Brayton) para realizar la conversión de energía. Esta turbomáquina es similar en tamaño a los turbocompresores para motores diésel marinos o de camiones grandes, y puede beneficiarse de los precios de las economías de escala. Los motores Brayton no necesitan refrigeración por rociado con agua y, de ese modo, se adaptan mejor a los ambientes secos del desierto.

El otro componente principal de una planta de energía CSP es un campo de espejos controlados, llamados heliostatos. Este campo tiene miles de metros cuadrados de heliostatos que concentran la energía solar en el receptor de la central eléctrica. El campo de heliostatos constituye una parte significativa del costo de una planta de CSP, y así llamó nuestra atención por oportunidades de reducción de costos.

Nuestro diseño del helióstato

El módulo reflector

Cada heliostato tenía un espejo de enfoque de 2m x 3m articulado en la parte superior de su marco. Se utilizó un diseño liviano. El módulo reflector del espejo estaba hecho de vidrio. Se realizaron pruebas de granizo para verificar el cumplimiento de los estándares de la industria.

El marco del helióstato y la base

Muchos marcos y bases de heliostatos existentes son estructuras sólidas montadas sobre una base de hormigón vertido en un sitio plano. Utilizan accionamientos de precisión y grandes actuadores para realizar apuntamientos rígidos. Para reducir el costo, nuestro diseño tenía un marco liviano y fácil de transportar. Fue sujetado por un anclaje de tierra. Montados en el bastidor había dos accionadores de cable que usaban pequeños motores baratos. El heliostato tenía una junta en U, que se articulaba en inclinación/balanceo.

El diseño del campo.

Cada uno de nuestros sistemas modulares de conversión de potencia del motor Brayton fue diseñado para producir una salida eléctrica planificada de 890kW por torre, lo que requeriría 2600kW de energía solar térmica proveniente de la apertura del receptor.

Utilizamos nuestro software de simulación de heliostatos ópticos (HOpS) para experimentar con diferentes configuraciones de campo.

Establecimos un tamaño de campo de 862 heliostatos alrededor de una torre de 44 m, cada heliostato es de aproximadamente 6m2. Los heliostatos están dispuestos en un patrón hexagonal, a una distancia horizontal máxima de 60 m desde la torre.

Consideramos que los heliostatos tienen la funcionalidad de apuntar a múltiples torres.

Requisitos de focalización del sistema de control

Para convertir la energía de manera eficiente, el motor Brayton requiere un receptor de cavidad de temperatura más alta que el receptor típico de una planta de vapor. Se deben producir cambios pequeños e infrecuentes en el flujo de calor a través de su intercambiador de calor para prolongar su vida útil.

Un receptor de mayor temperatura requiere una abertura más pequeña para reducir la pérdida de calor radiante. Para prolongar la vida útil del receptor de alta temperatura, el flujo debe distribuirse cuidadosamente dentro de la cavidad del receptor.

Sistema de detección y control

El sistema de control es capaz de controlar simultáneamente los puntos de luz desde múltiples heliostatos a un alto grado de precisión a un lugar deseado en un objetivo. Fue rastreado para compensar el movimiento del sol a través del cielo mientras corrige los efectos del viento constante. No es sensible a los efectos de los cambios de cimentación o la expansión térmica del marco.

Se utilizó un acelerómetro de 3 ejes montado en heliostato de bajo costo combinado con un sistema central de fotometría multiscópica para resolver las posiciones de puntos de luz individuales en el objetivo.

Mitigación del viento

El viento presenta un desafío de diseño particularmente difícil, especialmente cuando se trata de diseñar heliostatos más ligeros y de bajo costo. Las áreas de tierra grandes y planas donde es más probable que se construyan los heliostatos son también las áreas más propensas al viento sin restricciones.

Exploramos varias estrategias diferentes de mitigación del viento mediante el análisis y la experimentación.

Los heliostatos a lo largo del borde exterior de un campo protegen a los heliostatos en el medio de gran parte del impacto del viento. Además, las cercas de viento simples pueden reducir dramáticamente el impacto del viento en los heliostatos. [43]



Tecnología solar de alta concentración

La actividad del Grupo de Alta Concentración Solar (GACS) se centra fundamentalmente en los sistemas de Receptor Central. El despliegue comercial de plantas solares termoeléctricas de Receptor Central (STE-RC) se inició tímidamente en España, con la inauguración de PS10 (2007) y PS20 (2009). A nivel internacional se observa, a finales de 2010, un renovado interés por las plantas de foco puntual.
La “curva de aprendizaje” de la solar termoeléctrica de Receptor central se basaba en el ensayo de más de 10 instalaciones experimentales de receptor central en el mundo y una amplia variedad de componentes.
La experiencia acumulada ha servido para demostrar la viabilidad técnica del concepto y su capacidad para trabajar a altas temperaturas e integrarse en ciclos más eficientes de forma escalonada. También han demostrado que admiten fácilmente el funcionamiento híbrido en varias opciones y tienen el potencial de generar electricidad mediante el uso de almacenamiento térmico.
Las plantas de Receptor Central ofrecen por un lado mayores eficiencias totales de conversión y por otro una mayor diversidad de opciones de diseño, con menor experiencia acumulada en la implementación de cada tipología o componente. No obstante, el elevado coste de inversión aún constituye un obstáculo hacia el pleno aprovechamiento de su potencial a nivel comercial.
Este objetivo de reducir los costes de producción de electricidad está orientando los esfuerzos de I+D, por un lado a mejorar las opciones existentes y por otro a  demostrar la viabilidad de nuevas opciones de diseño.
Conscientes de la diversidad de opciones en competencia, sin que se tengan criterios claramente determinantes de elección, el GACS, además de participar en los primeros proyectos de demostración comercial de TRC, mantiene con carácter permanente una línea de I+D centrada en el desarrollo tecnológico de componentes y sistemas con el fin de generar información que ayude a reducir las incertidumbres y analizar la viabilidad técnica de las diferentes opciones. [44]

\end{document}


